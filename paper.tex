% This is paper.tex
\section{Introduction}
\textbf{Note:} If possible, this document should be printed in color.  However, I have left blue lines as solid and red lines as dashed for readability.

The Hecke Algebra (or Iwahori-Hecke Algebra) $\HH$ of a Weyl group $W$ is an object of significant interest, studied in representation theory, algebraic number theory, and combinatorics.

In 1979, Kazhdan and Lusztig \cite{ref:KaLu1,ref:KaLu2} gave a categorification of $\HH$, using a construction involving a complicated structure known as perverse sheaves on the flag variety.  The result became known as the Hecke category.
% If $K = \RR$, the resulting basis of $\HH$ is understood, and called the Kazhdan-Lusztig basis.  However, if $K$ is a field of some other characteristic $p$, the basis is unknown.
Soergel \cite{ref:Soe1,ref:Soe2} simplified this construction significantly in 1990 with an algebraic approach, using the aptly named Soergel bimodules.  The Soergel bimodules permit an easier understanding of the categorification of $\HH$; their advantage is that they are easier to define and are combinatorial in nature.
Furthermore, Soergel's construction extends the Hecke category to apply for arbitrary Coxeter systems $W$.

For a full treatise on the topic, we refer the reader to \cite{ref:Humphreys}.

Soergel bimodules can be described as summands of another class of bimodules, known as the \emph{Bott-Samelson bimodules}.
Now the morphisms of Bott-Samelson bimodules admit a combinatorial description.  In particular, \cite{ref:gr4all} gives a presentation for the category of these morphisms in terms of planar diagrammatics (i.e. diagrams in the plane).

This paper investigates some of the properties of these diagrammatics from a combinatorial perspective.   However, in this paper, we consider only the special case of an infinite dihedral group with generators $s$ and $t$ (as opposed to an arbitrary Coxeter group $W$).

Given a string of length $n$ composed of the letters $s$ and $t$, as well as a binary string of length $n$, one can construct such a diagram using an algorithm known as Libedinsky's light leaves.  The light leaves were derived by Libednisky in \cite{ref:Lib} and are detailed in Appendix~\ref{sec:prelim_lightleave}.  
These light leaves are interesting because they form a basis for a certain span of maps, as shown in \cite{ref:gr4all}.
We develop tools to understand the composition of two such maps given the original binary strings, with the eventual goal of deriving a combinatorial formula.
%The project deals with composing two maps formed in such a manner.
%The goal of the project is to describe the relationship between such a composition and the two original binary strings.

The preliminaries for the problem -- namely, the light leaves and their algebraic meaning -- are given in Appendix~\ref{sec:prelim_background}.  A minimal set of definitions used in this paper are given in Section~\ref{sec:prelim}.  The problem statement in its full technicality is given in Section~\ref{sec:probstate}.

The results begin by solving the special case of one color completely in Section~\ref{sec:res_onecolor}.  In Section~\ref{sec:res_count_maps}, we also present a formula for the number of maps between two given bimodules.  The most interesting results are in Sections~\ref{sec:res_alt_restrict} and \ref{sec:res_alt_compute}, where we investigate a special ``alternating'' case.  In particular, we show that the possible resulting structures are extremely limited, and provide a recursion based on these restrictions for that case.

\section{Preliminaries and Definitions}
\label{sec:prelim}
In Appendix~\ref{sec:prelim_background}, we provide the background behind the Bott-Samelson bimodules, and the correspondence between these maps and certain diagrams.  We then introduce Libedinsky's light leaves, a basis by which maps can be constructed using binary strings.

In Section~\ref{sec:prelim_def_notation}, we introduce several of our definitions which we use throughout the paper.  Next, in Section~\ref{sec:prelim_explain_poly_eval}, we describe how these maps can be represented as polynomials.  The actual problem statement is then deferred to Section~\ref{sec:probstate}.
\subsection{Definitions and Notations}
\label{sec:prelim_def_notation}
Take an infinite dihedral group $W=\left<s,t\mid s^2=t^2=1\right>$ and let $R = \RR[\alpha_s, \alpha_t]$ be the Coxeter ring as defined in Appendix~\ref{sec:prelim_background}, where $\alpha_s$ denotes reflection over the hyperplane normal to $s$.  
\subsubsection{Notations for Strings}
For each positive integer $n$, let us define for convenience the strings
\[ \SS_n \defeq \underbrace{s \dots s}_{\text{$n$ $s$'s}} \] 
and the ``alternating'' string
\[
	\AA_n \defeq
	\begin{cases}
		\underbrace{sts\dots s}_{\text{$n$ letters}} & \text{if } n \equiv 1 \pmod{2} \\
		\underbrace{sts\dots t}_{\text{$n$ letters}} & \text{if } n \equiv 0 \pmod{2}. 
	\end{cases}
\]
Similarly, let $\AA_n'$ denote $\AA_n$, except with the roles of $s$ and $t$ switched.  Note that $\AA_0 = \AA_0' = \varnothing$.

In general, a string of $s$'s and $t$'s will be referred to as an \emph{expression}.  In the case where there are no consecutive identical characters (in other words, the string is either $\AA_n$ or $\AA_n'$), we will refer to them as \emph{reduced expressions}.

Next, let $\BB_n$ denote the set of binary strings of length $n$.

Finally, given a string $\ul x$ (either a binary string or an expression) we will denote the $i\th$ character as $\ul x[i]$.  In general, underlined Roman letters denote strings.

\subsubsection{Definitions for Maps and Diagram}
Consider an expression $\ul r$.  Given a binary string $\ul b$, we will let $\MM_{\ul r}(\ul b)$ denote the map (or its associated diagram; we will refer to maps and their associated diagrams interchangeably) formed by the light leaves, as described in Appendix~\ref{sec:prelim_lightleave}.  We will refer to such structures as \emph{half-maps} or \emph{half-diagrams}.  Figure~\ref{fig:lightleaf_example} (in Appendix~\ref{sec:prelim_lightleave}) gives an example of such a half-map.

Additionally, let $\Top_{\ul r}(\ul b)$ denote the \emph{top} of the half-diagram $\MM_{\ul r}(\ul b)$, as described in Appendix~\ref{sec:prelim_lightleave}.  
Note that the top of any half-map is always a reduced expression.  In fact, one can verify that the top of a map is given explicitly by
\begin{equation}
	\Top_{\ul r}(\ul b) = \prod_{i=1}^n \left( \ul r[i] \right)^{\ul b[i]} \in W.
	\label{eq:top}
\end{equation}

\begin{figure}[ht]
	\centering
	\begin{asy}
	size(4cm);
	real h = 0.7;
	pen s = blue, t = red + dashed + 0.6;
	pen dot_s = blue, dot_t = red;
	int n = 7;

	picture one;
	draw(one, (0,0)--(0,h/2)..((0+3)/2.0,h*3)..(3,h/2)--(3,0), s);
	draw(one, (4,0)--(4,h/2)..((4+6)/2.0,h*2)..(6,h/2)--(6,0), t);
	draw(one, (1,0)--(1,h), t);
	dot(one, (1,h), dot_t);
	draw(one, (2,0)--(2,h), t);
	dot(one, (2,h), dot_t);
	draw(one, (5,0)--(5,h), s);
	dot(one, (5,h), dot_s);
	draw(one,(6,0)--(6,4*h), t);
	dot(one, (6, h/2), dot_t);

	picture two;
	draw(two, (0,0)--(0,h/2)..((0+3)/2.0,h*3)..(3,h/2)--(3,0), s);
	draw(two, (1,0)--(1,h), t);
	dot(two, (1,h), dot_t);
	draw(two, (2,0)--(2,h), t);
	dot(two, (2,h), dot_t);
	draw(two, (4,0)--(4,h), t);
	dot(two, (4,h), dot_t);
	draw(two, (5,0)--(5,h), s);
	dot(two, (5,h), dot_s);
	draw(two,(6,0)--(6,4*h), t);

	add(one); add(reflect((0,0),(1,0))*two);
	draw((-1,0)--(7,0));
	\end{asy}
	\caption{An example of composing two maps, $\MM_{sttstst}(1001100)$ and $\MM_{sttstst}(1001001)$.}
	\label{fig:example_compose}
\end{figure}

Two maps $\MM_{\ul r}(\ul a)$ and $\MM_{\ul r}(\ul b)$ with identical tops may be composed by juxtaposing $\MM_{\ul r}(\ul a)$ with a flipped copy of $\MM_{\ul r}(\ul b)$; an example of such a composition is given in Figure~\ref{fig:example_compose}.  We will denote this product by $\MM_{\ul r}(\ul a, \ul b)$.  If the two half-maps are not compatible (that is, $\Top_{\ul r}(\ul a) \neq \Top_{\ul r}(\ul b)$) then we will simply set $\MM_{\ul r}(\ul a, \ul b) = 0$.

Such structures will be called \emph{full-maps} or \emph{full-diagrams} for clarity.  If $\ul r$ is clear from context, we will abbreviate $\MM_{\ul r}$ as simply $\MM$ and $\Top_{\ul r}$ as $\Top$.

Recall that the light leaves are based off a sequence of vertices at the bottom boundary of a half-map.  We will call those vertices, whether on the bottom boundary of a half-map or the center of a full-map, the \emph{anchors} for that map, and the associated sequence of letters the \emph{base}.  

An example of a full-map with base $\AA_5$ is given in Figure~\ref{fig:example_full_map}.

\begin{figure}[ht]
	\centering
	\begin{asy}
		size(4cm);
		real h = 0.7;
		pen s = blue, t = red + dashed + 0.6;
		pen dot_s = blue, dot_t = red;
		int n = 5;

		picture one;
		draw(one, (0,0)--(0,h/2)..((0+2)/2.0,h*2)..(2,h/2)--(2,0), s);
		draw(one, (1,0)--(1,h), t);
		dot(one, (1,h), dot_t);
		draw(one, (3,0)--(3,h), t);
		dot(one, (3,h), dot_t);
		draw(one, (4,0)--(4,h), s);
		dot(one, (4,h), dot_s);

		picture two;
		draw(two, (1,0)--(1,h/2)..((1+3)/2.0,h*2)..(3,h/2)--(3,0), t);
		draw(two, (0,0)--(0,h/2)..((0+4)/2.0,h*4)..(4,h/2)--(4,0), s);
		draw(two, (2,0)--(2,h), s);
		dot(two, (2,h), dot_s);

		add(one); add(reflect((0,0),(1,0))*two);
		draw((-1,0)--(5,0));
	\end{asy}
	\caption{A full-map with base $ststs$, hence with five anchors.  The top is $\varnothing$.}
	\label{fig:example_full_map}
\end{figure}

\subsubsection{Nomenclature for Certain Structures in Diagrams}
Let us make a few convenient definitions for some recurring characters in our full-diagrams.

\begin{figure}[ht]
	\centering
	\begin{asy}
		size(5cm);
		real h = 0.7;
		pen s = blue, t = red + dashed + 0.6;
		pen dot_s = blue, dot_t = red;
		int n = 10;

		picture one;
		draw(one, (4,0)--(4,h/2)..((4+6)/2.0,h*2)..(6,h/2)--(6,0), s);
		draw(one, (3,0)--(3,h/2)..((3+7)/2.0,h*4)..(7,h/2)--(7,0), t);
		draw(one, (2,0)--(2,h/2)..((2+8)/2.0,h*6)..(8,h/2)--(8,0), s);
		draw(one, (1,0)--(1,h/2)..((1+9)/2.0,h*8)..(9,h/2)--(9,0), t);
		draw(one, (5,0)--(5,h), t);
		dot(one, (5,h), dot_t);
		draw(one,(0,0)--(0,9*h), s);

		picture two;
		draw(two, (2,0)--(2,h/2)..((2+4)/2.0,h*2)..(4,h/2)--(4,0), s);
		draw(two, (1,0)--(1,h/2)..((1+5)/2.0,h*4)..(5,h/2)--(5,0), t);
		draw(two, (0,0)--(0,h/2)..((0+6)/2.0,h*6)..(6,h/2)--(6,0), s);
		draw(two, (3,0)--(3,h), t);
		dot(two, (3,h), dot_t);
		draw(two, (7,0)--(7,h), t);
		dot(two, (7,h), dot_t);
		draw(two, (9,0)--(9,h), t);
		dot(two, (9,h), dot_t);
		draw(two,(8,0)--(8,7*h), s);

		add(one); add(reflect((0,0),(1,0))*two);
		draw((-1,0)--(10,0));
	\end{asy}
	\caption{Two red barbells with a very twisted fence, which creates two pastures.}
	\label{fig:def_barbell_fence}
\end{figure}

\begin{definition}
	A connected component which (i) is acyclic, and (ii) does not touch the top or bottom boundaries is called a \emph{barbell}.
\end{definition}
Notice that, by combining homotopy and contraction (see Appendix~\ref{sec:prelim_genrel}), every blue barbell is simply $\barbell{barbell_blue}$.  Similarly, every red barbell is simply $\barbell{barbell_red}$.

\begin{definition}
	A \emph{fence} is a contiguous path which runs from the top of the boundary to the bottom of the boundary (i.e. paths between the labelled vertices).  The diagram is divided by these fences into \emph{pastures}.
\end{definition}
\begin{definition}
	A component is called \emph{attached} if it is connected to a fence.  Otherwise, it is \emph{free}.  By convention, any attached components belong to the pasture to the left of the fence to which it is attached.
\end{definition}
An example of a fence complete with two barbells is given in Figure~\ref{fig:def_barbell_fence}.

Next, we name some recurrent characters in our work.

% We remind the reader that we are working modulo lower terms; any map which contains a ``half fence'' (i.e. a connected component which touches a boundary exactly once) is not considered.

\begin{figure}[ht]
	\centering
	\begin{asy}
		size(4cm);
		real h = 0.7;
		pen s = blue, t = red + dashed + 0.6;
		pen dot_s = blue, dot_t = red;
		int n = 4;

		picture one;
		draw(one, (0,0)--(0,h/2)..((0+3)/2.0,h*3)..(3,h/2)--(3,0), s);
		draw(one, (1,0)--(1,h), t);
		dot(one, (1,h), dot_t);
		draw(one, (2,0)--(2,h), t);
		dot(one, (2,h), dot_t);

		picture two;
		draw(two, (0,0)--(0,h/2)..((0+3)/2.0,h*3)..(3,h/2)--(3,0), s);
		draw(two, (1,0)--(1,h), t);
		dot(two, (1,h), dot_t);
		draw(two, (2,0)--(2,h), t);
		dot(two, (2,h), dot_t);

		add(one); add(reflect((0,0),(1,0))*two);
		draw((-1,0)--(4,0));
	\end{asy}
	\caption{A blue bubble with two barbells inside it.}
	\label{fig:def_bubble}
\end{figure}

\begin{definition}
	A \emph{bubble} is a bounded face along with any components contained within the face.  The \emph{color} of the bubble is the color of the edges which form the bounded face.  
\end{definition}
\begin{definition}
	We say that the bubble $B$ \emph{holds} a component $c$ if $c$ is inside $B$, but there is no other bubble $B'$ inside $B$ which contains $c$.  
\end{definition}

An example of a bubble is given in Figure~\ref{fig:def_bubble}.

\begin{figure}[ht]
	\centering
	\begin{asy}
		size(6cm);
		real h = 0.7;
		pen s = blue, t = red + dashed + 0.6;
		pen dot_s = blue, dot_t = red;
		int n = 7;

		picture one;
		draw(one, (0,0)--(0,h/2)..((0+2)/2.0,h*2)..(2,h/2)--(2,0), s);
		draw(one, (2,0)--(2,h/2)..((2+4)/2.0,h*2)..(4,h/2)--(4,0), s);
		draw(one, (4,0)--(4,h/2)..((4+6)/2.0,h*2)..(6,h/2)--(6,0), s);
		draw(one, (1,0)--(1,h), t);
		dot(one, (1,h), dot_t);
		draw(one, (3,0)--(3,h), t);
		dot(one, (3,h), dot_t);
		draw(one, (5,0)--(5,h), t);
		dot(one, (5,h), dot_t);
		dot(one, (2, h/2), dot_s);
		dot(one, (4, h/2), dot_s);

		picture two;
		draw(two, (0,0)--(0,h/2)..((0+2)/2.0,h*2)..(2,h/2)--(2,0), s);
		draw(two, (2,0)--(2,h/2)..((2+4)/2.0,h*2)..(4,h/2)--(4,0), s);
		draw(two, (4,0)--(4,h/2)..((4+6)/2.0,h*2)..(6,h/2)--(6,0), s);
		draw(two, (1,0)--(1,h), t);
		dot(two, (1,h), dot_t);
		draw(two, (3,0)--(3,h), t);
		dot(two, (3,h), dot_t);
		draw(two, (5,0)--(5,h), t);
		dot(two, (5,h), dot_t);
		dot(two, (2, h/2), dot_s);
		dot(two, (4, h/2), dot_s);

		add(one); add(reflect((0,0),(1,0))*two);
		draw((-1,0)--(7,0));
	\end{asy}
	\caption{A caterpillar made of three bubbles}
	\label{fig:def_caterpillar}
\end{figure}

\begin{definition}
	A \emph{caterpillar} consists of a free component and the contents of any bubbles formed by its edges.  A caterpillar \emph{holds} a component if some bubble of the caterpillar holds it.
\end{definition}
An example of a caterpillar is given in Figure~\ref{fig:def_caterpillar}.  Notice that any barbell or bubble is a special case of a caterpillar.


\subsection{Representations of Maps as Polynomials}
\label{sec:prelim_explain_poly_eval}

In order to study these maps as algebraic structures, we wish to consider full-maps as polynomials in $\RR[x,y][\alpha_s,\alpha_t]$.  Hence, we establish the following conventions:
\begin{enumerate}
	\ii The map $x \mapsto fx$ will be abbreviated as $f$ from this point forward.  
	\par In particular, the map $\barbell{barbell_blue}$ is $\alpha_s$.  This follows from a straightforward computation.  Furthermore, one can check that, say, $\barbell{barbell_blue}\barbell{barbell_blue}\barbell{barbell_red}\barbell{alpha_blue} = \alpha_s^2 \alpha_t$.  
	Please note that $x \mapsto xf$ does not get a similar abbreviation.  So, $\barbell{alpha_blue}\barbell{barbell_blue} \neq \alpha_s$.
	\ii We will work \emph{modulo lower terms}.  That is, any map which contains a connected component that touches any boundary only once (total) is considered zero.  So, for example, we have $\barbell{break_blue} = 0$.
\end{enumerate}
Henceforth, we will refer to diagrams and their associated polynomials interchangeably.

One can verify that, using the barbell-forcing rules, we can recursively move all barbells to the left, and any ``broken'' walls are reduced to zero, or are contracted.  Hence, any full-diagram can be reduced to linear combinations of maps of the form $f \mapsto \alpha_s^m \alpha_t^n f$.  In other words, every full-diagram can be represented as a polynomial in $\RR[x,y][\alpha_s,\alpha_t]$.

In fact, we can even show that, with this new convention, that the following identity holds for every $f$:
\begin{equation}
	\barbell{alpha_blue} f = \partial_s(f) \barbell{break_blue} + s(f) \barbell{alpha_blue}
	\label{eq:barbell_break}
\end{equation}
where $\partial_s$ is the Demazure operator defined in Appendix~\ref{sec:prelim_genrel}.

For reference, Appendix~\ref{sec:ststs_matrix} contains a complete table of all maps of the form $\MM_{\AA_5} (\ul a, \ul b)$.

\subsubsection{Quantum Numbers}
\label{sec:quantum}
A particularly nice sequence of polynomials can be defined recursively.  They are termed \emph{two-color quantum numbers}.
\begin{definition}[Quantum Numbers]
	For each integer $n \ge 0$, define two polynomials $[n]_x, [n]_y \in \ZZ[x,y]$ recursively by $[0]_x = [0]_y = 0$, $[1]_x = [1]_y = 1$, $[2]_x = x$, $[2]_y = y$, and
	\begin{align*}
		[2]_x[n]_y = [n+1]_x - [n-1]_x \\
		[2]_y[n]_x = [n+1]_y - [n-1]_y
	\end{align*}
	Finally, $[-n]_x = -[n]_x$ and $[-n]_y = -[n]_y$ for each $n \ge 0$.
\end{definition}
The quantities $[n]_x$ and $[n]_y$ are closely related.  In particular, $[2k+1]_x = [2k+1]_y$ and $[2]_x[2k]_y = [2]_y[2k]_x$ for all integers $k$.  This follows immediately by induction.

The quantum numbers are relevant to the problem because of the identity
\begin{equation}
	s([n]_x\alpha_s + [n-1]_y\alpha_t) = [n]_x\alpha_s + [n+1]_y\alpha_t
	\label{eq:quantum}
\end{equation}
and its analogous form for $t$.  One can verify that \eqref{eq:quantum} immediately follows from definitions.  This gives us a robust way to express the results of action on $\alpha_s$ or $\alpha_t$ by elements of $W$.  For example, $ststst(\alpha_s) = ststst([1]_x\alpha_s + [0]_y\alpha_t) = [7]_x\alpha_s + [6]_y\alpha_t$.

This motivates us to make the following definition, which will be used in Section~\ref{sec:res_alt_compute}.
\begin{definition}
	For any reduced expression $\ul r$, we define the quantum polynomial $Q_{\ul r}$ by :
	\[
		Q_{\AA_n} = 
		\begin{cases}
			[n+1]_x \alpha_s + [n]_y \alpha_t & \text{if $n$ is odd} \\
			[n]_x \alpha_s + [n+1]_y \alpha_t & \text{if $n$ is even}
		\end{cases}
	\]
	when $\ul r = \AA_n$, and otherwise,
	\[
		Q_{\AA_n'} =
		\begin{cases}
			[n]_x \alpha_s + [n+1]_y \alpha_t & \text{if $n$ is even} \\
			[n+1]_x \alpha_s + [n]_y \alpha_t & \text{if $n$ is odd}.
		\end{cases}
	\]
\end{definition}

The significance is that, if $\ul r$ is a reduced expression of length $n+1$, then \[ \MM_{\ul r} ( \underbrace{11\dots11}_{\text{$n$ 1's}}0, \underbrace{11\dots11}_{\text{$n$ 1's}}0 ) = Q_{\ul r}. \]
An illustration of such a full-map is given in Figure~\ref{fig:push_quantum}.
This follows from \eqref{eq:barbell_break}.  

\begin{figure}[ht]
	\centering
	\begin{asy}
		size(4cm);
		real h = 0.7;
		pen s = blue, t = red + dashed + 0.6;
		pen dot_s = blue, dot_t = red;
		int n = 5;

		picture one;
		draw(one, (4,0)--(4,h), s);
		dot(one, (4,h), dot_s);
		draw(one,(0,0)--(0,3*h), s);
		draw(one,(1,0)--(1,3*h), t);
		draw(one,(2,0)--(2,3*h), s);
		draw(one,(3,0)--(3,3*h), t);

		picture two;
		draw(two, (4,0)--(4,h), s);
		dot(two, (4,h), dot_s);
		draw(two,(0,0)--(0,3*h), s);
		draw(two,(1,0)--(1,3*h), t);
		draw(two,(2,0)--(2,3*h), s);
		draw(two,(3,0)--(3,3*h), t);

		add(one); add(reflect((0,0),(1,0))*two);
		draw((-1,0)--(5,0));
	\end{asy}
	\caption{$\MM(11110,11110)$.}
	\label{fig:push_quantum}
\end{figure}



\section{Problem Statement}
\label{sec:probstate}
\begin{definition}
	A map is \emph{idempotent} if multiplying it with itself returns the original map.  Two idempotents are orthogonal if their product is zero.
\end{definition}
Fix an expression $\ul r$.  If we can find two maps $A$ and $B$ for which $AB$ is the identity, then $(BA)^2=BABA=BA$ will be an idempotent.  More generally, however, if $AB = c$ for some scalar $c$, then \[ \left( \frac{BA}{c} \right)\left( \frac{BA}{c} \right) = \frac{BABA}{c^2} = \frac{BA}{c} \] is also an idempotent.  In other words, any two maps with nonzero product can be used to construct an idempotent.

However, if we are interested in finding sets of pairwise orthogonal idempotents, then this becomes a linear algebra problem in which we require a matrix of all possible products.  The construction of such a set of idempotents is closely related to the decomposition of these bimodules, which in turn gives information about the Soergel bimodules\footnote{Actually, the Soergel bimodules may be defined as the images of idempotents in the endomorphism rings of the Bott-Samelson bimodules.}.

This motivates the main objective of study.
\begin{ques*}
	For a fixed $\ul r$ and two binary sequences $\ul a$ and $\ul b$ with $\Top_{\ul r}(\ul a) = \Top_{\ul r}(\ul b)$, find a formula for $\MM_{\ul r}(\ul a, \ul b)$ in terms of $\ul r$, $\ul a$, and $\ul b$.
\end{ques*}
Here, we are interested in computing the polynomial which corresponds to this map.

\section{A Complete Characterization of the One-Color Case}
\label{sec:res_onecolor}
In this section we provide a complete characterization for $\MM_{\SS_n}(\ul a, \ul b)$.  % The proof is effectively trivial, but requires some rather irritating details and definitions in order to be clear.

First, we need a criteria to determine the top.  Fortunately, this is not hard.
\begin{definition}
	For a binary string $\ul b$, let $\pi_1(\ul b)$ denote the number of $1$'s in $\ul b$.
\end{definition}
\begin{proposition}
	For any $\ul b \in \BB_n$, \[
		\Top_{\SS_n}(\ul b) =
		\begin{cases}
			\varnothing & \text{if } \pi_1(\ul b) \equiv 0 \pmod{2} \\
			s & \text{if } \pi_1(\ul b) \equiv 1 \pmod{2}.
		\end{cases}
		\]
	\label{prop:s_top}
\end{proposition}
\begin{proof}
	Straightforward induction on $n$.  
\end{proof}

Subsequently, we make the following definition for this section only.
\begin{definition}
	For a binary string $\ul b$, define the \emph{partial-sum string} of $\ul b$, denoted $\ul b^\ast$, as follows:
	\[
		\ul b^\ast [i] = 
		\begin{cases}
			1 & \text{if } \ul b[1] + \ul b[2] + \dots + \ul b[i] \equiv 1 \pmod{2} \\
			0 & \text{otherwise}.
		\end{cases}
	\]
\end{definition}
This lets us state our main result for this section. 
\begin{theorem}
	If $\ul a, \ul b \in \BB_n$ obey $\Top_{\SS_n}(\ul a) = \Top_{\SS_n}(\ul b)$, then
	\[
		\MM_{\SS_n} \left( \ul a, \ul b \right)
		=
		\begin{cases}
			0 & \text{if }\ \exists 1 \le i \le n-1:\ \ul a^\ast[i] = \ul b^\ast[i] = 1 \\
			\alpha_s^{n - \pi_1(\ul a^\ast) - \pi_1(\ul b^\ast) + \ul a^\ast[n]} & \text{otherwise}.
		\end{cases}
	\]
	\label{thm:s_eval}
\end{theorem}
The proof is essentially only a matter of finding loops and counting connected components.
As such, it is deferred to Appendix~\ref{sec:s_proof}.

\section{Counting Maps}
\label{sec:res_count_maps}
We give a formula for the number of binary strings $\ul b$ such that $\Top_{\ul r}(\ul b) = \ul T$ for each expression $\ul T$ and $\ul r$.  

\begin{definition}
	Let $R(\ul r)$ denote $\ul r$ with any consecutive blocks of $s$'s replaced by a single $s$, and similarly for $t$.
\end{definition}
For example, $R(sststttts) = ststs$.

\begin{proposition}
	Let $\ul r$ be an expression of length $\ell$, and suppose $R(\ul r) = \AA_n$.
	\begin{enumerate}[(i)]
		\ii If $n=2k+1$ is odd, then for each $0 \le m \le k$, there are $2^{\ell-n} \binom{2k}{m+k}$ binary strings $\ul b$ yielding each of $\AA_{2m+1}$, $\AA_{2m}$, $\AA_{2m-1}'$, and $\AA_{2m}'$ as $\Top_{\ul r}(\ul b)$.
		\ii If $n=2k$ is even, then for each $0 \le m \le k-1$, there are $2^{\ell-n} \binom{2k-1}{m+k}$ binary strings $\ul b$ yielding each of $\AA_{2m+1}$, $\AA_{2m+2}$, $\AA_{2m}'$, and $\AA_{2m+1}'$ as $\Top_{\ul r}(\ul b)$.
	\end{enumerate}
	\label{thm:count_maps}
\end{proposition}
Of course, if $\ul r$ begins with the character $t$, then one can simply reverse the roles of $s$ and $t$ in the above theorem to obtain an analogous formula.

The main idea of the proof is to reduce the problem to the case when $\ell = n$, and then apply induction on $n$.  We fill in the details below.

\begin{proof}[Proof of Proposition \ref{thm:count_maps}]
	First we will resolve the case where $\ul r = \AA_n$, so that $\ell = n$ and $2^{\ell-n} = 1$.  This follows by induction on $n$.  The main idea is that, for the binary strings of length $n$ giving a top $T$, we consider those that end with $1$ and those that end with $0$.  Using \eqref{eq:top}, we may then write the new quantity in terms of the number of strings of length $n-1$ with a different top.
	% After deleting the last bit, the former type have a top which differs by one character according to $\AA_n[n]$, while the latter have the same top.
	We simply check that in all eight cases, the binomial coefficients add up to give the new binomial coefficients (via Pascal's identity).  The computation is left to the reader.

	To finish the case $\ell \neq n$, we note that we can treat each consecutive block of $s$'s of length $m$ as a single $s$, where there are $2^{m-1}$ ways to get a $1$ and $2^{m-1}$ ways to get a $0$.  (This follows from Proposition \ref{prop:s_top}.)  This reduces to the previous case $\ul r = \AA_n$, but with some extra factors of $2$; aggregating all of them yields the factor of $2^{\ell-n}$, as desired.
\end{proof}

\section{Colorful Lemmata for the Alternating Case}
\label{sec:res_alt_restrict}
We now turn our attention to the case of $\ul r = \AA_n$.

We now present three results which show that the possible outputs of a map with base $\AA_n$ are extremely restricted.  
\begin{lemma}[Bubble Lemma]
	Consider a bubble in a full-diagram with base $\AA_n$.  The bubble holds exactly one caterpillar, whose color is opposite that of the bubble.
	\label{thm:bubble}
\end{lemma}
\begin{lemma}[Caterpillar Lemma]
	Any caterpillar in a full-diagram with base $\AA_n$ evaluates in the form $x^my^n\alpha_s$ if it is blue, or $x^my^n\alpha_t$ if it is red.
	\label{thm:caterpillar}
\end{lemma}
\begin{lemma}[Pasture Lemma]
	Consider a full-diagram with $N+1$ pastures (numbered $0$, $1$, \dots, $N$) with base $\AA_n$ and top $T$.  Suppose further that $N \ge 1$ (or equivalently, $T \neq \varnothing$).  Then:
	\begin{enumerate}[(i)]
		\ii Pastures $1$ through $N-1$ contain either nothing or an attached component.
		\ii Pasture $N$ contains nothing if $T[N] = \AA_n[n]$, and contains a single caterpillar otherwise.  The caterpillar is blue if $\AA_n[n]$ is $s$, and red otherwise.
	\end{enumerate}
	\label{thm:pasture}
\end{lemma}

Included in Figure~\ref{fig:large_alt_example} is an example of a map with base $\AA_{27}$.   The Bubble Lemma and Pasture Lemma can be readily observed in this example.

\begin{figure}[ht]
	\centering
	\begin{asy}
		size(17cm);
		real h = 0.7;
		pen s = blue, t = red + dashed + 0.6;
		pen dot_s = blue, dot_t = red;
		int n = 27;

		picture one;
		draw(one, (0,0)--(0,h/2)..((0+2)/2.0,h*2)..(2,h/2)--(2,0), s);
		draw(one, (5,0)--(5,h/2)..((5+7)/2.0,h*2)..(7,h/2)--(7,0), t);
		draw(one, (7,0)--(7,h/2)..((7+9)/2.0,h*2)..(9,h/2)--(9,0), t);
		draw(one, (12,0)--(12,h/2)..((12+14)/2.0,h*2)..(14,h/2)--(14,0), s);
		draw(one, (11,0)--(11,h/2)..((11+15)/2.0,h*4)..(15,h/2)--(15,0), t);
		draw(one, (17,0)--(17,h/2)..((17+19)/2.0,h*2)..(19,h/2)--(19,0), t);
		draw(one, (21,0)--(21,h/2)..((21+23)/2.0,h*2)..(23,h/2)--(23,0), t);
		draw(one, (23,0)--(23,h/2)..((23+25)/2.0,h*2)..(25,h/2)--(25,0), t);
		draw(one, (20,0)--(20,h/2)..((20+26)/2.0,h*6)..(26,h/2)--(26,0), s);
		draw(one, (1,0)--(1,h), t);
		dot(one, (1,h), dot_t);
		draw(one, (3,0)--(3,h), t);
		dot(one, (3,h), dot_t);
		draw(one, (6,0)--(6,h), s);
		dot(one, (6,h), dot_s);
		draw(one, (8,0)--(8,h), s);
		dot(one, (8,h), dot_s);
		draw(one, (13,0)--(13,h), t);
		dot(one, (13,h), dot_t);
		draw(one, (18,0)--(18,h), s);
		dot(one, (18,h), dot_s);
		draw(one, (22,0)--(22,h), s);
		dot(one, (22,h), dot_s);
		draw(one, (24,0)--(24,h), s);
		dot(one, (24,h), dot_s);
		draw(one,(4,0)--(4,7*h), s);
		draw(one,(9,0)--(9,7*h), t);
		draw(one,(10,0)--(10,7*h), s);
		draw(one,(15,0)--(15,7*h), t);
		draw(one,(16,0)--(16,7*h), s);
		draw(one,(19,0)--(19,7*h), t);
		dot(one, (7, h/2), dot_t);
		dot(one, (23, h/2), dot_t);
		dot(one, (9, h/2), dot_t);
		dot(one, (15, h/2), dot_t);
		dot(one, (19, h/2), dot_t);

		picture two;
		draw(two, (5,0)--(5,h/2)..((5+7)/2.0,h*2)..(7,h/2)--(7,0), t);
		draw(two, (7,0)--(7,h/2)..((7+9)/2.0,h*2)..(9,h/2)--(9,0), t);
		draw(two, (12,0)--(12,h/2)..((12+14)/2.0,h*2)..(14,h/2)--(14,0), s);
		draw(two, (14,0)--(14,h/2)..((14+16)/2.0,h*2)..(16,h/2)--(16,0), s);
		draw(two, (17,0)--(17,h/2)..((17+19)/2.0,h*2)..(19,h/2)--(19,0), t);
		draw(two, (21,0)--(21,h/2)..((21+23)/2.0,h*2)..(23,h/2)--(23,0), t);
		draw(two, (23,0)--(23,h/2)..((23+25)/2.0,h*2)..(25,h/2)--(25,0), t);
		draw(two, (20,0)--(20,h/2)..((20+26)/2.0,h*6)..(26,h/2)--(26,0), s);
		draw(two, (0,0)--(0,h), s);
		dot(two, (0,h), dot_s);
		draw(two, (1,0)--(1,h), t);
		dot(two, (1,h), dot_t);
		draw(two, (2,0)--(2,h), s);
		dot(two, (2,h), dot_s);
		draw(two, (3,0)--(3,h), t);
		dot(two, (3,h), dot_t);
		draw(two, (6,0)--(6,h), s);
		dot(two, (6,h), dot_s);
		draw(two, (8,0)--(8,h), s);
		dot(two, (8,h), dot_s);
		draw(two, (13,0)--(13,h), t);
		dot(two, (13,h), dot_t);
		draw(two, (15,0)--(15,h), t);
		dot(two, (15,h), dot_t);
		draw(two, (18,0)--(18,h), s);
		dot(two, (18,h), dot_s);
		draw(two, (22,0)--(22,h), s);
		dot(two, (22,h), dot_s);
		draw(two, (24,0)--(24,h), s);
		dot(two, (24,h), dot_s);
		draw(two,(4,0)--(4,7*h), s);
		draw(two,(9,0)--(9,7*h), t);
		draw(two,(10,0)--(10,7*h), s);
		draw(two,(11,0)--(11,7*h), t);
		draw(two,(16,0)--(16,7*h), s);
		draw(two,(19,0)--(19,7*h), t);
		dot(two, (7, h/2), dot_t);
		dot(two, (14, h/2), dot_s);
		dot(two, (23, h/2), dot_t);
		dot(two, (9, h/2), dot_t);
		dot(two, (16, h/2), dot_s);
		dot(two, (19, h/2), dot_t);

		add(one); add(reflect((0,0),(1,0))*two);
		draw((-1,0)--(27,0));
	\end{asy}
	\caption{A large map with base $\AA_{27}$.}
	\label{fig:large_alt_example}
\end{figure}

The Bubble Lemma is the key combinatorial observation, which hinges on the fact that any two anchors with the same color have a third anchor of the other color between them.  The other two lemmas follow directly from it.
The details of the proofs are given in Appendix~\ref{sec:lemmata_proof}.

\section{Algebraic Computations for the Alternating Case}
\label{sec:res_alt_compute}
We now turn to using the tools developed in Section~\ref{sec:res_alt_restrict} in order to compute polynomials. First, we add some structure to these special cases.

\subsection{Restrictions and Definitions for the Final Output}
Let us now endow our maps with base $\AA_n$ with some additional structure.
\begin{theorem}
	For any binary strings $\ul a, \ul b \in \BB_n$, if $\MM_{\AA_n} (\ul a, \ul b)$ is nonzero and has top $T$, then for some nonnegative integers $m_x$, $m_y$, $m_s$, and $m_t$, we have 
	\[
		\MM_{\AA_n} (\ul a, \ul b) =
		x^{m_x}y^{m_y}\alpha_s^{m_s}\alpha_t^{m_t} \cdot
		\begin{cases}
			1 & \text{if $T$ and $\AA_n$ have the same last character} \\
			Q_T & \text{otherwise}.
		\end{cases}
	\]
	\label{thm:alt_struct}
\end{theorem}
Here, $Q_T$ is the quantum polynomial defined in Section~\ref{sec:quantum}.
\begin{proof}
	This is an immediate consequence of the Caterpillar Lemma and the Pasture Lemma.  Everything in the first pasture is a product of monomials by the Caterpillar Lemma.  If $T$ and $\AA_n$ agree, then the last pasture is empty and hence we obtain the first case.  Otherwise, there is a single caterpillar of color opposite to $T$, and hence after evaluating that caterpillar, we obtain some monomial times a single barbell.  This generates the final $Q_T$ term after the barbell is pushed through all the fences; the broken fences are all zero because they are lower terms.
\end{proof}

This motivates the following definition, which is relevant for Proposition~\ref{thm:recurse}.
\begin{definition}
	If a nonzero map $M$ with base $\AA_n$ is presented as in Theorem~\ref{thm:alt_struct}, then the term $x^{m_x}y^{m_y}\alpha_s^{m_s}\alpha_t^{m_t}$ will be called the \emph{fluff} of $M$, and the polynomial (either $1$ or $Q_T$) will be called the \emph{face} of $M$. 
\end{definition}
\begin{remark*}
	According to Theorem~\ref{thm:alt_struct}, a nonzero map $\MM_{\AA_n}$ has face $1$ and only if its top has the same last character as $\AA_n$.
\end{remark*}

\subsection{A Partial Recursion}
In this section we present a limited recursion for computing $\MM_{\AA_n}(\ul a, \ul b)$.  

\begin{proposition}
	Let $n \ge 2$ be an odd integer and consider two binary strings $\ul a, \ul b \in \BB_n$ such that $T = \Top_{\AA_n}(\ul a) = \Top_{\AA_n}(\ul b)$.  
	Let $\ul a'$ and $\ul b'$ be the first $n-1$ bits of $\ul a$ and $\ul b$, respectively. 
	Suppose $M_1 = \MM_{\AA_{n-1}} \left(\ul a', \ul b'\right)$ is a nonzero map with fluff $F$.
	\begin{enumerate}[(i)]
		\ii If $\ul a[n] = \ul b[n] = 0$, then
		\[ \MM_{\AA_n} (\ul a, \ul b)
			=
			\begin{cases}
				\alpha_s M_1 & \text{if $T = \varnothing$} \\ % just another barbell
				-xF & \text{if $T$ ends in $s$} \\ % closes a caterpillar like 0|
				 Q_T M_1 & \text{if $T$ ends in $t$}. \\ % just another barbell, but push it all the way through nonempty T
			\end{cases}
		\]
		\ii If $\ul a[n] = \ul b[n] = 1$, then
		\[
			\MM_{\AA_n} (\ul a, \ul b)
			=
			\begin{cases}
				-x\alpha_sF & \text{if $T = \varnothing$} \\ % destructive 1, closes a caterpillar O
				M_1 & \text{if $T$ ends in $s$} \\ % constructive 1
				-xF Q_T & \text{if $T$ ends in $t$}. % destructive 1
			\end{cases}
		\]
	\end{enumerate}
	Analogous statements hold when $n$ is an even integer, with the roles of $x$ and $y$ interchanged, as well as the roles of $s$ and $t$.
	\label{thm:recurse}
\end{proposition}
The recursion arises because, using the lemmata, we are able to see exactly what happens in the rightmost pasture if a pair of $1$'s or a pair of $0$'s is added.  In short, either a barbell or fence is added on the far right, or a caterpillar is enclosed in a bubble; each of these leads to one of the cases above.  The full computation is given in Appendix~\ref{sec:recurse_proof}.

Notice that this recursion is sufficient to calculate $\MM_{\AA_n}(\ul b, \ul b)$.


\section{Conclusion}
The overarching goal of the paper was to compute $\MM_{\ul r} (\ul a, \ul b)$ in terms of $\ul r$, $\ul a$, and $\ul b$.  In this paper, we solved the case where $\ul r = \SS_n$ completely, and showed that the case where $\ul r = \AA_n$ is significantly restricted.  We also provide a partial recurrence for $\AA_n$.  One obvious next direction would be to attempt to complete the recursion for $\AA_n$ so that $\MM_{\AA_n} (\ul a, \ul b)$ can be determined completely.

What about the original problem of arbitrary $\ul r$?  It was originally hoped that, by understanding $\AA_n$ and $\SS_n$, one could derive the general formula (perhaps by viewing strings of $s$'s in $\ul r$ as a single $s$ with a formula for $\SS_n$, and then aggregating all this information with a formula for $\AA_n$).  Unfortunately, the lemmata of Section~\ref{sec:res_alt_restrict} show that in fact, maps with base $\AA_n$ are quite restricted.  So, either the results of Section~\ref{sec:res_alt_compute} will have to be generalized significantly (in the context of maps outside those of base $\AA_n$), or an entirely new approach will be required.

Yet another possible direction is to search for similar results when the order of $st$ in $W$ is finite, or even if $W$ is an arbitrary Coxeter group.

\section{Acknowledgments} 
I would like to thank my head mentor Dr. Tanya Khovanova, MIT, for providing individual discussion with me, and the MIT Math Department for making my project possible.
I also wish to thank my tutor Mr. Antoni Rangachev, Northeastern University, for his invaluable advice in preparing both this paper and my final presentation.
I also thank the Center for Excellence in Education and the Research Science Institute, as well as the Massachusetts Institute of Technology, for generously providing the facilities for the research.

I would also like to thank my sponsors:
\begin{inparaenum}[]
	\ii Ms. Alexa Margalith, The Leonetti\slash O'Connell Family Foundation;
	\ii Mr. Arvind Parthasarathi, YarcData;
	\ii Mr. Samuel Chen and Ms. Kathy Chang;
	\ii Dr. Robert E. Curry;
	\ii Mr. and Mrs. William Hellman;
	\ii Mr. Ronald Houhauser;
	\ii Ms. Chienlan Hsu-Hoffman; and
	\ii Mr. and Mrs. George Keiter.
\end{inparaenum}

Finally, I offer my sincerest gratitude to my mentor Mr. Francisco Unda, MIT, for daily mentorship, and Professor Benjamin Elias, MIT, for providing the project and for often meeting with me personally to discuss it.
That sounded horribly fake, but I really mean it.  You guys are the best.  Thank you for a great project.
