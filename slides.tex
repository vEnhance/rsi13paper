% This is slides.tex. Compile me using ``make slides.pdf''.

\documentclass[pdf]{beamer}
\mode<presentation>{}                % change the theme here

%% preamble {{{1
\usepackage{rsislidepacks}
\usepackage{colortbl}
\title[Maps in $\mathcal B_{\text{BS}}$]{Diagrammatic Computation of Morphisms Between Bott-Samelson Bimodules via Libedinsky's Light Leaves}
\subtitle[RSI 2013]{Research Science Institute 2013}
\author[Evan Chen]{Evan Chen \\ Under the Direction of \\ Francisco Unda \\ Massachusetts Institute of Technology}

\usetheme{UNLTheme}
\setbeamercovered{dynamic}

\usepackage{enumerate}
\usepackage{hyperref}
\usepackage{asymptote}


% For code
\usepackage{listings}
\lstset{basicstyle=\ttfamily,
	numbers=left,
	numbersep=5pt,
	numberstyle=\tiny,
	keywordstyle=\bfseries,
	title=\lstname,
	showstringspaces=false,
	frame=single}


% Small commands
\newcommand{\myqed}{\textsc{Q.e.d.}}
\newcommand{\cbrt}[1]{\sqrt[3]{#1}}
\newcommand{\floor}[1]{\left\lfloor #1 \right\rfloor}
\newcommand{\ceiling}[1]{\left\lceil #1 \right\rceil}
\newcommand{\mailto}[1]{\href{mailto:#1}{#1}}
\renewcommand{\iff}{\Leftrightarrow}
\renewcommand{\implies}{\Rightarrow}
\newcommand{\hrulebar}{
  \par\hspace{\fill}\rule{0.95\linewidth}{.7pt}\hspace{\fill}
  \par\nointerlineskip \vspace{\baselineskip}
}
\def\half{\frac{1}{2}}

%More commands and math operators
\DeclareMathOperator{\cis}{cis}
\DeclareMathOperator{\lcm}{lcm}

%Convenient Environments
\newenvironment{soln}{\begin{proof}[Solution]}{\end{proof}}
\newenvironment{parlist}{\begin{inparaenum}[(i)]}{\end{inparaenum}}
\newenvironment{gobble}{\setbox\z@\vbox\bgroup}{\egroup}

%Inequalities
\newcommand{\cycsum}{\sum_{\text{cyc}}}
\newcommand{\symsum}{\sum_{\text{sym}}}
\newcommand{\cycprod}{\prod_{\text{cyc}}}
\newcommand{\symprod}{\prod_{\text{sym}}}

%From H113 "Introduction to Abstract Algebra" at UC Berkeley
\def\CC{\mathbb C}
\def\FF{\mathbb F}
\def\NN{\mathbb N}
\def\QQ{\mathbb Q}
\def\RR{\mathbb R}
\def\ZZ{\mathbb Z}
\newcommand{\normal}{\trianglelefteq}
\newcommand{\charin}{\text{ char }}
\DeclareMathOperator{\sign}{sign}
\DeclareMathOperator{\Aut}{Aut}
\DeclareMathOperator{\Inn}{Inn}
\DeclareMathOperator{\Syl}{Syl}

%From Kiran Kedlaya's "Geometry Unbound"
\def\abs#1{\lvert #1 \rvert}
\def\norm#1{\lVert #1 \rVert}
\def\dang{\measuredangle} %% Directed angle
\def\line#1{\overleftrightarrow{#1}}
\def\ray#1{\overrightarrow{#1}} 
\def\seg#1{\overline{#1}}
\def\arc#1{\wideparen{#1}}

%From M275 "Topology" at SJSU
\newcommand{\id}{\text{id}}
\newcommand{\taking}[1]{\stackrel{#1}{\longrightarrow}}
\newcommand{\inv}{^{-1}}

%From M170 "Introduction to Graph Theory" at SJSU
\DeclareMathOperator{\diam}{diam}
\DeclareMathOperator{\ord}{ord}
\newcommand{\defeq}{\stackrel{\text{def}}{=}}

%From the USAMO .tex filse
\def\st{^{\text{st}}}
\def\nd{^{\text{nd}}}
\def\rd{^{\text{rd}}}
\def\th{^{\text{th}}}
\def\dg{^\circ}
\def\be{\begin{enumerate}}
\def\bee{\begin{enumerate} \ii}
\def\ee{\end{enumerate}}
\def\bi{\begin{itemize}}
\def\bii{\begin{itemize} \ii}
\def\ei{\end{itemize}}
\def\ii{\item}

%Asy commands
\begin{asydef}
	import olympiad;
	import cse5;
	pointpen = black;
	pathpen = black;
	pathfontpen = black;
	anglepen = black;
	anglefontpen = black;
\end{asydef}

% }}}
% RSI Specific macros
\theoremstyle{definition}
\newtheorem{op}{Operation}
\newcommand{\dobarbell}[2]{
	\vcenter{\hbox{%
		\includegraphics[scale=#1]{barbell/#2.pdf}
	}}
}
\newcommand{\barbell}[1]{
	\mathchoice%
	{\dobarbell{1.6}{#1}}
	{\dobarbell{1.2}{#1}}
	{\dobarbell{0.9}{#1}}
	{\dobarbell{0.9}{#1}}
}

\def\DD{\mathcal D}
\def\ul#1{\underline{#1}}
\begin{document}

\begin{frame}
	\maketitle
\end{frame}


\begin{frame}[fragile]
	\frametitle{Description of the Diagrams}
	This project involves computations of certain maps which can be represented with diagrams.
\end{frame}

\begin{frame}[fragile]
	Diagrams are \alert{planar graphs} such that:
	\begin{enumerate}
		\ii Vertices may lie on top/bottom boundaries but not on sides.
		\ii Each vertex has degree $1$ or $3$.
		\ii Connected components colored either $s$ (blue) or $t$ (red).
		\ii Graphs may be deformed continuously.
	\end{enumerate}
	% At each end, the vertices give a sequence of $s$ (blue) and $t$ (red).
	% The vertices on the boundary are by convention not explicitly shown, but are nonetheless labelled $s$ or $t$ for blue or red, respectively.
	\begin{figure}[ht]
		\centering
		\begin{asy}
		size(2.718cm);
		real xmax=7;
		real ymax=5;
		draw( (xmax,ymax)--(xmax,-ymax)--(-xmax,-ymax)--(-xmax,ymax)--cycle );
		pair apex = (0,2);
		path arc = (5,-5)..(2,0)..apex..(-2,0)..(-5,-5);
		draw(arc, blue);
		dot(apex, blue);
		draw(apex--(0,ymax), blue);
		draw(-apex--(0,-ymax), red);
		dot(-apex, red);
		label("$s$", (0,ymax), dir(90));
		label("$t$", (0,-ymax), dir(270));
		label("$s$", (-5,-ymax), dir(270));
		label("$s$", (5,-ymax), dir(270));
		\end{asy}
		\caption{An example of a diagram a possible diagram.}
	\end{figure}
\end{frame}

%\begin{frame}
%	\frametitle{Permitted Operations on the Graphs}
%	\begin{enumerate}
%		\addtocounter{enumi}{-1}
%		\ii (Isotropy) Diagrams can be continuously deformed.
%		\ii (Associativity) $\barbell{assoc_horiz} = \barbell{assoc_vert}$.
%		\ii (Contraction) $\barbell{contract_left} = \barbell{contract_right} = \barbell{alpha_blue}$.
%		\ii (The Needle) $\barbell{needle} = \barbell{zero} = 0$.
%		\ii (Barbell-Forcing)
%		\begin{enumerate}[(a)]
%			\ii $\barbell{barbell_blue}\barbell{alpha_blue} + \barbell{alpha_blue} \barbell{barbell_blue} = 2 \barbell{break_blue}$, and the similar equation for red.
%			\ii $\barbell{alpha_red}\barbell{barbell_blue} = -x\barbell{break_red} + \barbell{barbell_blue}\barbell{alpha_red} + x \barbell{barbell_red}\barbell{alpha_blue}$.
%			\ii $\barbell{alpha_blue}\barbell{barbell_red} = -y\barbell{break_blue} + \barbell{barbell_red}\barbell{alpha_blue} + y \barbell{barbell_blue}\barbell{alpha_red}$.
%		\end{enumerate}
%	\end{enumerate}
%\end{frame}

\begin{frame}[fragile]
	\frametitle{Light Leaves}
	\begin{itemize}
		\ii Fix an expression $\ul r$ of $s$'s and $t$'s with length $n$.
		\ii Take any binary string $\ul b$ of length $n$.
		\ii Each choice of $\ul b$ will correspond to a diagram, based on a certain set of rules.
	\end{itemize}
	\begin{figure}[ht]
		\centering
		\begin{asy}
		size(10cm);
		real h = 0.7;
		pen s = blue, t = red;
		int n = 14;
		draw(currentpicture, (0,0)--(0,h/2)..((0+2)/2.0,h*2)..(2,h/2)--(2,0), s);
		draw(currentpicture, (2,0)--(2,h/2)..((2+3)/2.0,h*1)..(3,h/2)--(3,0), s);
		draw(currentpicture, (5,0)--(5,h/2)..((5+6)/2.0,h*1)..(6,h/2)--(6,0), s);
		draw(currentpicture, (4,0)--(4,h/2)..((4+8)/2.0,h*4)..(8,h/2)--(8,0), t);
		draw(currentpicture, (8,0)--(8,h/2)..((8+9)/2.0,h*1)..(9,h/2)--(9,0), t);
		draw(currentpicture, (11,0)--(11,h/2)..((11+13)/2.0,h*2)..(13,h/2)--(13,0), t);
		draw(currentpicture, (1,0)--(1,h), t);dot(currentpicture, (1,h), t);
		draw(currentpicture, (7,0)--(7,h), s);dot(currentpicture, (7,h), s);
		draw(currentpicture, (12,0)--(12,h), s);dot(currentpicture, (12,h), s);
		draw(currentpicture,(9,0)--(9,5*h), t);
		draw(currentpicture,(10,0)--(10,5*h), s);
		dot(currentpicture, (2, h/2), s);
		dot(currentpicture, (8, h/2), t);
		dot(currentpicture, (9, h/2), t);
		label(currentpicture, "1", (0,-1.5h), dir(90));
		label(currentpicture, "s", (0,-0.8*h), dir(90));
		label(currentpicture, "0", (1,-1.5h), dir(90));
		label(currentpicture, "t", (1,-0.8*h), dir(90));
		label(currentpicture, "0", (2,-1.5h), dir(90));
		label(currentpicture, "s", (2,-0.8*h), dir(90));
		label(currentpicture, "1", (3,-1.5h), dir(90));
		label(currentpicture, "s", (3,-0.8*h), dir(90));
		label(currentpicture, "1", (4,-1.5h), dir(90));
		label(currentpicture, "t", (4,-0.8*h), dir(90));
		label(currentpicture, "1", (5,-1.5h), dir(90));
		label(currentpicture, "s", (5,-0.8*h), dir(90));
		label(currentpicture, "1", (6,-1.5h), dir(90));
		label(currentpicture, "s", (6,-0.8*h), dir(90));
		label(currentpicture, "0", (7,-1.5h), dir(90));
		label(currentpicture, "s", (7,-0.8*h), dir(90));
		label(currentpicture, "0", (8,-1.5h), dir(90));
		label(currentpicture, "t", (8,-0.8*h), dir(90));
		label(currentpicture, "0", (9,-1.5h), dir(90));
		label(currentpicture, "t", (9,-0.8*h), dir(90));
		label(currentpicture, "1", (10,-1.5h), dir(90));
		label(currentpicture, "s", (10,-0.8*h), dir(90));
		label(currentpicture, "1", (11,-1.5h), dir(90));
		label(currentpicture, "t", (11,-0.8*h), dir(90));
		label(currentpicture, "0", (12,-1.5h), dir(90));
		label(currentpicture, "s", (12,-0.8*h), dir(90));
		label(currentpicture, "1", (13,-1.5h), dir(90));
		label(currentpicture, "t", (13,-0.8*h), dir(90));
		\end{asy}
		\caption{An example of light leaves}
	\end{figure}
\end{frame}

\begin{frame}[fragile]
	\frametitle{Problem Statement}
	Two maps can be composed by placing one on top of the other.  Then, operations may be applied to simplify the diagram.

	The goal of the project is to \alert{compute the product} (or otherwise deduce information about it) using \alert{only the binary strings}.
	\begin{figure}[ht]
		\centering
		\begin{asy}
		size(4cm);
		real h = 0.7;
		pen s = blue, t = red;
		int n = 7;

		picture one;
		draw(one, (0,0)--(0,h/2)..((0+3)/2.0,h*3)..(3,h/2)--(3,0), s);
		draw(one, (4,0)--(4,h/2)..((4+6)/2.0,h*2)..(6,h/2)--(6,0), t);
		draw(one, (1,0)--(1,h), t);
		dot(one, (1,h), t);
		draw(one, (2,0)--(2,h), t);
		dot(one, (2,h), t);
		draw(one, (5,0)--(5,h), s);
		dot(one, (5,h), s);
		draw(one,(6,0)--(6,4*h), t);
		dot(one, (6, h/2), t);

		picture two;
		draw(two, (0,0)--(0,h/2)..((0+3)/2.0,h*3)..(3,h/2)--(3,0), s);
		draw(two, (1,0)--(1,h), t);
		dot(two, (1,h), t);
		draw(two, (2,0)--(2,h), t);
		dot(two, (2,h), t);
		draw(two, (4,0)--(4,h), t);
		dot(two, (4,h), t);
		draw(two, (5,0)--(5,h), s);
		dot(two, (5,h), s);
		draw(two,(6,0)--(6,4*h), t);

		add(one); add(reflect((0,0),(1,0))*two);
		draw((-1,0)--(7,0));
		\end{asy}
		\caption{An example of composing two maps.}
	\end{figure}
\end{frame}

\begin{frame}
	\frametitle{Special Case}
	\begin{enumerate}
		\ii $\ul r = \underbrace{s\dots s}_{\text{$n$ $s$'s}}$.
		\pause \par This case turns out to be trivial.
		\pause
		\ii $\ul r = \underbrace{stst\dots}_{\text{$n$ characters}}$.
		\par This case has some interesting properties but is not as trivial.
	\end{enumerate}
\end{frame}


\begin{frame}[fragile]
	\frametitle{Barbells, Fences and Pastures}
	\begin{definition}
		\begin{itemize}
			\ii A \emph{barbell} is a tree not touching the boundaries.
			\ii A \emph{fence} is a contiguous path which from top to bottom.
			\ii The diagram is divided by these fences into \emph{pastures}.
		\end{itemize}
	\end{definition}
	\begin{figure}[ht]
		\centering
		\begin{asy}
			size(3cm);
			real h = 0.7;
			pen s = blue, t = red + dashed + 0.6;
			pen dot_s = blue, dot_t = red;
			int n = 10;

			picture one;
			draw(one, (4,0)--(4,h/2)..((4+6)/2.0,h*2)..(6,h/2)--(6,0), s);
			draw(one, (3,0)--(3,h/2)..((3+7)/2.0,h*4)..(7,h/2)--(7,0), t);
			draw(one, (2,0)--(2,h/2)..((2+8)/2.0,h*6)..(8,h/2)--(8,0), s);
			draw(one, (1,0)--(1,h/2)..((1+9)/2.0,h*8)..(9,h/2)--(9,0), t);
			draw(one, (5,0)--(5,h), t);
			dot(one, (5,h), dot_t);
			draw(one,(0,0)--(0,9*h), s);

			picture two;
			draw(two, (2,0)--(2,h/2)..((2+4)/2.0,h*2)..(4,h/2)--(4,0), s);
			draw(two, (1,0)--(1,h/2)..((1+5)/2.0,h*4)..(5,h/2)--(5,0), t);
			draw(two, (0,0)--(0,h/2)..((0+6)/2.0,h*6)..(6,h/2)--(6,0), s);
			draw(two, (3,0)--(3,h), t);
			dot(two, (3,h), dot_t);
			draw(two, (7,0)--(7,h), t);
			dot(two, (7,h), dot_t);
			draw(two, (9,0)--(9,h), t);
			dot(two, (9,h), dot_t);
			draw(two,(8,0)--(8,7*h), s);

			add(one); add(reflect((0,0),(1,0))*two);
			draw((-1,0)--(10,0));
		\end{asy}
		\caption{Two blue barbells with a very twisted fence, which creates two pastures.}
	\end{figure}
\end{frame}

\begin{frame}[fragile]
	\frametitle{Bubbles and Caterpillars}
	\begin{definition}
		\begin{itemize}
			\ii A \emph{bubble} is a bounded face in a diagram.
			\ii A \emph{caterpillar} is a connected set of bubbles.
		\end{itemize}
	\end{definition}

\begin{figure}[ht]
	\centering
	\begin{asy}
		size(8cm);
		real h = 0.7;
		pen s = blue, t = red + dashed + 0.6;
		pen dot_s = blue, dot_t = red;
		int n = 11;

		picture one;
		draw(one, (0,0)--(0,h/2)..((0+3)/2.0,h*3)..(3,h/2)--(3,0), s);
		draw(one, (4,0)--(4,h/2)..((4+6)/2.0,h*2)..(6,h/2)--(6,0), s);
		draw(one, (6,0)--(6,h/2)..((6+8)/2.0,h*2)..(8,h/2)--(8,0), s);
		draw(one, (8,0)--(8,h/2)..((8+10)/2.0,h*2)..(10,h/2)--(10,0), s);
		draw(one, (1,0)--(1,h), t);
		dot(one, (1,h), dot_t);
		draw(one, (2,0)--(2,h), t);
		dot(one, (2,h), dot_t);
		draw(one, (5,0)--(5,h), t);
		dot(one, (5,h), dot_t);
		draw(one, (7,0)--(7,h), t);
		dot(one, (7,h), dot_t);
		draw(one, (9,0)--(9,h), t);
		dot(one, (9,h), dot_t);
		dot(one, (6, h/2), dot_s);
		dot(one, (8, h/2), dot_s);

		picture two;
		draw(two, (0,0)--(0,h/2)..((0+3)/2.0,h*3)..(3,h/2)--(3,0), s);
		draw(two, (4,0)--(4,h/2)..((4+6)/2.0,h*2)..(6,h/2)--(6,0), s);
		draw(two, (6,0)--(6,h/2)..((6+8)/2.0,h*2)..(8,h/2)--(8,0), s);
		draw(two, (8,0)--(8,h/2)..((8+10)/2.0,h*2)..(10,h/2)--(10,0), s);
		draw(two, (1,0)--(1,h), t);
		dot(two, (1,h), dot_t);
		draw(two, (2,0)--(2,h), t);
		dot(two, (2,h), dot_t);
		draw(two, (5,0)--(5,h), t);
		dot(two, (5,h), dot_t);
		draw(two, (7,0)--(7,h), t);
		dot(two, (7,h), dot_t);
		draw(two, (9,0)--(9,h), t);
		dot(two, (9,h), dot_t);
		dot(two, (6, h/2), dot_s);
		dot(two, (8, h/2), dot_s);

		add(one); add(reflect((0,0),(1,0))*two);
		draw((-1,0)--(11,0));
	\end{asy}
	\caption{A blue bubble with two red barbells, and a caterpillar with three bubbles.}
\end{figure}
\end{frame}

\begin{frame}[fragile]
	\frametitle{Bubble Lemma}
	Suppose $\ul r = stst\dots$.

	\begin{lemma}[Bubble Lemma]
		Every bubble contains either a single caterpillar or a single barbell.
	\end{lemma}
	\begin{figure}[ht]
		\centering
		\begin{asy}
		size(3cm);
		real h = 0.7;
		pen s = blue, t = red + dashed + 0.6;
		pen dot_s = blue, dot_t = red;
		int n = 7;

		picture one;
		draw(one, (1,0)--(1,h/2)..((1+3)/2.0,h*2)..(3,h/2)--(3,0), t);
		draw(one, (3,0)--(3,h/2)..((3+5)/2.0,h*2)..(5,h/2)--(5,0), t);
		draw(one, (0,0)--(0,h/2)..((0+6)/2.0,h*6)..(6,h/2)--(6,0), s);
		draw(one, (2,0)--(2,h), s);
		dot(one, (2,h), dot_s);
		draw(one, (4,0)--(4,h), s);
		dot(one, (4,h), dot_s);
		dot(one, (3, h/2), dot_t);

		picture two;
		draw(two, (1,0)--(1,h/2)..((1+3)/2.0,h*2)..(3,h/2)--(3,0), t);
		draw(two, (3,0)--(3,h/2)..((3+5)/2.0,h*2)..(5,h/2)--(5,0), t);
		draw(two, (0,0)--(0,h/2)..((0+6)/2.0,h*6)..(6,h/2)--(6,0), s);
		draw(two, (2,0)--(2,h), s);
		dot(two, (2,h), dot_s);
		draw(two, (4,0)--(4,h), s);
		dot(two, (4,h), dot_s);
		dot(two, (3, h/2), dot_t);

		add(one); add(reflect((0,0),(1,0))*two);
		draw((-1,0)--(7,0));
		\end{asy}
		\caption{A bubble with a caterpillar inside it}
	\end{figure}
\end{frame}

\begin{frame}[fragile]

	\begin{theorem}[Pasture Theorem]
		 If pastures are labelled $0, 1, \dots, N$ then:
		\begin{itemize}
			\ii Pastures $1$ through $N-1$ must contain caterpillars attached to fences.
			\ii Pasture $N$ is either empty or contains a single caterpillar.
		\end{itemize}
	\end{theorem}

	\begin{figure}[ht]
		\centering
		\begin{asy}
			size(3cm);
real h = 0.7;
pen s = blue, t = red + dashed + 0.6;
pen dot_s = blue, dot_t = red;
int n = 9;

picture one;
draw(one, (2,0)--(2,h/2)..((2+4)/2.0,h*2)..(4,h/2)--(4,0), s);
draw(one, (1,0)--(1,h/2)..((1+5)/2.0,h*4)..(5,h/2)--(5,0), t);
draw(one, (6,0)--(6,h/2)..((6+8)/2.0,h*2)..(8,h/2)--(8,0), s);
draw(one, (3,0)--(3,h), t);
dot(one, (3,h), dot_t);
draw(one, (7,0)--(7,h), t);
dot(one, (7,h), dot_t);
draw(one,(0,0)--(0,5*h), s);
draw(one,(5,0)--(5,5*h), t);
dot(one, (5, h/2), dot_t);

picture two;
draw(two, (2,0)--(2,h/2)..((2+4)/2.0,h*2)..(4,h/2)--(4,0), s);
draw(two, (1,0)--(1,h/2)..((1+5)/2.0,h*4)..(5,h/2)--(5,0), t);
draw(two, (6,0)--(6,h/2)..((6+8)/2.0,h*2)..(8,h/2)--(8,0), s);
draw(two, (3,0)--(3,h), t);
dot(two, (3,h), dot_t);
draw(two, (7,0)--(7,h), t);
dot(two, (7,h), dot_t);
draw(two,(0,0)--(0,5*h), s);
draw(two,(5,0)--(5,5*h), t);
dot(two, (5, h/2), dot_t);

add(one); add(reflect((0,0),(1,0))*two);
draw((-1,0)--(9,0));
		\end{asy}
		\caption{An attached caterpillar and a single caterpillar (each are singletons.)}
	\end{figure}

\end{frame}

\begin{frame}
	\frametitle{Implications}
	\[ \text{Bubble Lemma} + \text{Pasture Theorem} = \quad :{)} \]
\end{frame}

\begin{frame}
	\frametitle{Acknowledgements}
	Thank you to the following individuals and organizations:
	\begin{itemize}
		\ii Mr. Francisco Unda for daily mentorship.
		\ii Prof. Ben Elias for providing the project and meeting to discuss it.
		\ii Dr. Tanya Khovanova and MIT Math for organizing Math RSI.
		\ii CEE and MIT for hosting the RSI 2013.
		\ii Antoni Rangachev for his advice both in this paper and presentation.
	\end{itemize}
\end{frame}



\end{document}
