% This is slides.tex. Compile me using ``make slides.pdf''.

\documentclass[pdf]{beamer}
\mode<presentation>{}                % change the theme here

%% preamble {{{1
\usepackage{rsislidepacks}
\usepackage{colortbl}
\title[Maps in $\mathcal B_{\text{BS}}$]{Diagrammatic Computation of Morphisms Between Bott-Samelson Bimodules via Libedinsky's Light Leaves}
\subtitle[RSI 2013]{Research Science Institute 2013}
\author{Evan Chen}

\usetheme{UNLTheme}
\setbeamercovered{dynamic}

\usepackage{enumerate}
\usepackage{hyperref}
\usepackage{asymptote}


% For code
\usepackage{listings}
\lstset{basicstyle=\ttfamily,
	numbers=left,
	numbersep=5pt,
	numberstyle=\tiny,
	keywordstyle=\bfseries,
	title=\lstname,
	showstringspaces=false,
	frame=single}


% Small commands
\newcommand{\myqed}{\textsc{Q.e.d.}}
\newcommand{\cbrt}[1]{\sqrt[3]{#1}}
\newcommand{\floor}[1]{\left\lfloor #1 \right\rfloor}
\newcommand{\ceiling}[1]{\left\lceil #1 \right\rceil}
\newcommand{\mailto}[1]{\href{mailto:#1}{#1}}
\renewcommand{\iff}{\Leftrightarrow}
\renewcommand{\implies}{\Rightarrow}
\newcommand{\hrulebar}{
  \par\hspace{\fill}\rule{0.95\linewidth}{.7pt}\hspace{\fill}
  \par\nointerlineskip \vspace{\baselineskip}
}
\def\half{\frac{1}{2}}

%More commands and math operators
\DeclareMathOperator{\cis}{cis}
\DeclareMathOperator{\lcm}{lcm}

%Convenient Environments
\newenvironment{soln}{\begin{proof}[Solution]}{\end{proof}}
\newenvironment{parlist}{\begin{inparaenum}[(i)]}{\end{inparaenum}}
\newenvironment{gobble}{\setbox\z@\vbox\bgroup}{\egroup}

%Inequalities
\newcommand{\cycsum}{\sum_{\text{cyc}}}
\newcommand{\symsum}{\sum_{\text{sym}}}
\newcommand{\cycprod}{\prod_{\text{cyc}}}
\newcommand{\symprod}{\prod_{\text{sym}}}

%From H113 "Introduction to Abstract Algebra" at UC Berkeley
\def\CC{\mathbb C}
\def\FF{\mathbb F}
\def\NN{\mathbb N}
\def\QQ{\mathbb Q}
\def\RR{\mathbb R}
\def\ZZ{\mathbb Z}
\newcommand{\normal}{\trianglelefteq}
\newcommand{\charin}{\text{ char }}
\DeclareMathOperator{\sign}{sign}
\DeclareMathOperator{\Aut}{Aut}
\DeclareMathOperator{\Inn}{Inn}
\DeclareMathOperator{\Syl}{Syl}

%From Kiran Kedlaya's "Geometry Unbound"
\def\abs#1{\lvert #1 \rvert}
\def\norm#1{\lVert #1 \rVert}
\def\dang{\measuredangle} %% Directed angle
\def\line#1{\overleftrightarrow{#1}}
\def\ray#1{\overrightarrow{#1}} 
\def\seg#1{\overline{#1}}
\def\arc#1{\wideparen{#1}}

%From M275 "Topology" at SJSU
\newcommand{\id}{\text{id}}
\newcommand{\taking}[1]{\stackrel{#1}{\longrightarrow}}
\newcommand{\inv}{^{-1}}

%From M170 "Introduction to Graph Theory" at SJSU
\DeclareMathOperator{\diam}{diam}
\DeclareMathOperator{\ord}{ord}
\newcommand{\defeq}{\stackrel{\text{def}}{=}}

%From the USAMO .tex filse
\def\st{^{\text{st}}}
\def\nd{^{\text{nd}}}
\def\rd{^{\text{rd}}}
\def\th{^{\text{th}}}
\def\dg{^\circ}
\def\be{\begin{enumerate}}
\def\bee{\begin{enumerate} \ii}
\def\ee{\end{enumerate}}
\def\bi{\begin{itemize}}
\def\bii{\begin{itemize} \ii}
\def\ei{\end{itemize}}
\def\ii{\item}

%Asy commands
\begin{asydef}
	import olympiad;
	import cse5;
	pointpen = black;
	pathpen = black;
	pathfontpen = black;
	anglepen = black;
	anglefontpen = black;
\end{asydef}

% }}}
% RSI Specific macros
\theoremstyle{definition}
\newtheorem{op}{Operation}
\newcommand{\dobarbell}[2]{
	\vcenter{\hbox{%
		\includegraphics[scale=#1]{barbell/#2.pdf}
	}}
}
\newcommand{\barbell}[1]{
	\mathchoice%
	{\dobarbell{1.6}{#1}}
	{\dobarbell{1.2}{#1}}
	{\dobarbell{0.9}{#1}}
	{\dobarbell{0.9}{#1}}
}

\def\DD{\mathcal D}
\def\ul#1{\underline{#1}}
\begin{document}

\begin{frame}
	\maketitle
\end{frame}

\begin{frame}
	\frametitle{Algebraic Nonsense}
	The focus of the project is computations involving graphical representation of certain maps, largely \alert{ignoring the algebraic background}.
	\par
	But if you're curious\dots
	\begin{itemize}
		\ii Once upon a time, in a land far far away, there was a Coxeter ring $R = \RR[\alpha_s, \alpha_t]$ acted on by the corresponding Coxeter system $(\mathcal W, \mathcal S)$ with $\mathcal S = \left\{ s,t \right\}$.
		\ii Then there were Bott-Samelson bimodules given by the tensor product $B_s = R \otimes_{R^s} R$, where $R^s$ is $s$-invariant.
		\ii There are maps between these modules, which can be represented graphically.
	\end{itemize}
\end{frame}

\begin{frame}[fragile]
	\frametitle{Description of the Diagrams}
	%We consider an category $\DD$ whose elements are linear combinations $\sum c_i \square$ of the diagrams described below.  The coefficients $c_i$ belong to the ring $\ZZ[x,y]$.

	Diagrams are \alert{planar graphs} such that:
	\begin{enumerate}
		\ii Vertices may lie on top/bottom but not on sides.
		\ii Each vertex has degree $1$ or $3$.
		\ii Connected components colored either $s$ (blue) or $t$ (red).
	\end{enumerate}
	% At each end, the vertices give a sequence of $s$ (blue) and $t$ (red).
	% The vertices on the boundary are by convention not explicitly shown, but are nonetheless labelled $s$ or $t$ for blue or red, respectively.
	\begin{figure}[ht]
		\centering
		\begin{asy}
		size(3cm);
		real xmax=7;
		real ymax=5;
		draw( (xmax,ymax)--(xmax,-ymax)--(-xmax,-ymax)--(-xmax,ymax)--cycle );
		pair apex = (0,2);
		path arc = (5,-5)..(2,0)..apex..(-2,0)..(-5,-5);
		draw(arc, blue);
		dot(apex, blue);
		draw(apex--(0,ymax), blue);
		draw(-apex--(0,-ymax), red);
		dot(-apex, red);
		label("$s$", (0,ymax), dir(90));
		label("$t$", (0,-ymax), dir(270));
		label("$s$", (-5,-ymax), dir(270));
		label("$s$", (5,-ymax), dir(270));
		\end{asy}
		\caption{An example of a diagram in $\DD$.}
	\end{figure}
\end{frame}

\begin{frame}
	\frametitle{Permitted Operations on the Graphs}
	\begin{enumerate}
		\addtocounter{enumi}{-1}
		\ii (Isotropy) Diagrams can be continuously deformed.
		\ii (Associativity) $\barbell{assoc_horiz} = \barbell{assoc_vert}$.
		\ii (Contraction) $\barbell{contract_left} = \barbell{contract_right} = \barbell{alpha_blue}$.
		\ii (The Needle) $\barbell{needle} = \barbell{zero} = 0$.
		\ii (Barbell-Forcing)
		\begin{enumerate}[(a)]
			\ii $\barbell{barbell_blue}\barbell{alpha_blue} + \barbell{alpha_blue} \barbell{barbell_blue} = 2 \barbell{break_blue}$, and the similar equation for red.
			\ii $\barbell{alpha_red}\barbell{barbell_blue} = -x\barbell{break_red} + \barbell{barbell_blue}\barbell{alpha_red} + x \barbell{barbell_red}\barbell{alpha_blue}$.
			\ii $\barbell{alpha_blue}\barbell{barbell_red} = -y\barbell{break_blue} + \barbell{barbell_red}\barbell{alpha_blue} + y \barbell{barbell_blue}\barbell{alpha_red}$.
		\end{enumerate}
	\end{enumerate}
\end{frame}

\begin{frame}[fragile]
	\frametitle{Light Leaves}
	\only<1>{
	\begin{itemize}
		\ii Fix an expression $\ul r$ of $s$'s and $t$'s with length $n$.
		\ii Take any binary string $\ul b$ of length $n$.
		\ii Each choice of $\ul b$ will correspond to a diagram, based on a certain set of rules.
	\end{itemize}
	}	
	\only<2>{
	Read $\ul b$ from left to right.  Mark bits as up or down.

	\begin{minipage}[t]{0.40\textwidth}
		For a $1$:
		\begin{itemize}
			\ii If OK, draw arc, mark both down.
			\ii Else, mark up.
		\end{itemize}
	\end{minipage}
	\begin{minipage}[t]{0.50\textwidth}
		For a $0$:
		\begin{itemize}
			\ii If OK, draw an arc, mark prev.\ down, this up.
			\ii Else, create ``stub''.
		\end{itemize}
	\end{minipage}
	%Finally, join any bits which are up at the end of the process to the opposite end.  
	}
	\begin{figure}[ht]
		\centering
		\begin{asy}
		size(10cm);
		real h = 0.7;
		pen s = blue, t = red;
		int n = 14;
		draw(currentpicture, (0,0)--(0,h/2)..((0+2)/2.0,h*2)..(2,h/2)--(2,0), s);
		draw(currentpicture, (2,0)--(2,h/2)..((2+3)/2.0,h*1)..(3,h/2)--(3,0), s);
		draw(currentpicture, (5,0)--(5,h/2)..((5+6)/2.0,h*1)..(6,h/2)--(6,0), s);
		draw(currentpicture, (4,0)--(4,h/2)..((4+8)/2.0,h*4)..(8,h/2)--(8,0), t);
		draw(currentpicture, (8,0)--(8,h/2)..((8+9)/2.0,h*1)..(9,h/2)--(9,0), t);
		draw(currentpicture, (11,0)--(11,h/2)..((11+13)/2.0,h*2)..(13,h/2)--(13,0), t);
		draw(currentpicture, (1,0)--(1,h), t);dot(currentpicture, (1,h), t);
		draw(currentpicture, (7,0)--(7,h), s);dot(currentpicture, (7,h), s);
		draw(currentpicture, (12,0)--(12,h), s);dot(currentpicture, (12,h), s);
		draw(currentpicture,(9,0)--(9,5*h), t);
		draw(currentpicture,(10,0)--(10,5*h), s);
		dot(currentpicture, (2, h/2), s);
		dot(currentpicture, (8, h/2), t);
		dot(currentpicture, (9, h/2), t);
		label(currentpicture, "1", (0,-1.5h), dir(90));
		label(currentpicture, "s", (0,-0.8*h), dir(90));
		label(currentpicture, "0", (1,-1.5h), dir(90));
		label(currentpicture, "t", (1,-0.8*h), dir(90));
		label(currentpicture, "0", (2,-1.5h), dir(90));
		label(currentpicture, "s", (2,-0.8*h), dir(90));
		label(currentpicture, "1", (3,-1.5h), dir(90));
		label(currentpicture, "s", (3,-0.8*h), dir(90));
		label(currentpicture, "1", (4,-1.5h), dir(90));
		label(currentpicture, "t", (4,-0.8*h), dir(90));
		label(currentpicture, "1", (5,-1.5h), dir(90));
		label(currentpicture, "s", (5,-0.8*h), dir(90));
		label(currentpicture, "1", (6,-1.5h), dir(90));
		label(currentpicture, "s", (6,-0.8*h), dir(90));
		label(currentpicture, "0", (7,-1.5h), dir(90));
		label(currentpicture, "s", (7,-0.8*h), dir(90));
		label(currentpicture, "0", (8,-1.5h), dir(90));
		label(currentpicture, "t", (8,-0.8*h), dir(90));
		label(currentpicture, "0", (9,-1.5h), dir(90));
		label(currentpicture, "t", (9,-0.8*h), dir(90));
		label(currentpicture, "1", (10,-1.5h), dir(90));
		label(currentpicture, "s", (10,-0.8*h), dir(90));
		label(currentpicture, "1", (11,-1.5h), dir(90));
		label(currentpicture, "t", (11,-0.8*h), dir(90));
		label(currentpicture, "0", (12,-1.5h), dir(90));
		label(currentpicture, "s", (12,-0.8*h), dir(90));
		label(currentpicture, "1", (13,-1.5h), dir(90));
		label(currentpicture, "t", (13,-0.8*h), dir(90));
		\end{asy}
		\caption{An example of light leaves}
	\end{figure}
\end{frame}

\begin{frame}[fragile]
	\frametitle{Problem Statement}
	Two maps can be composed by placing one on top of the other.  Then, operations may be applied to simplify the diagram.

	The goal of the project is to \alert{compute the product} (or otherwise deduce information about it) using \alert{only the binary strings}.
	\begin{figure}[ht]
		\centering
		\begin{asy}
		size(4cm);
		real h = 0.7;
		pen s = blue, t = red;
		int n = 7;

		picture one;
		draw(one, (0,0)--(0,h/2)..((0+3)/2.0,h*3)..(3,h/2)--(3,0), s);
		draw(one, (4,0)--(4,h/2)..((4+6)/2.0,h*2)..(6,h/2)--(6,0), t);
		draw(one, (1,0)--(1,h), t);
		dot(one, (1,h), t);
		draw(one, (2,0)--(2,h), t);
		dot(one, (2,h), t);
		draw(one, (5,0)--(5,h), s);
		dot(one, (5,h), s);
		draw(one,(6,0)--(6,4*h), t);
		dot(one, (6, h/2), t);

		picture two;
		draw(two, (0,0)--(0,h/2)..((0+3)/2.0,h*3)..(3,h/2)--(3,0), s);
		draw(two, (1,0)--(1,h), t);
		dot(two, (1,h), t);
		draw(two, (2,0)--(2,h), t);
		dot(two, (2,h), t);
		draw(two, (4,0)--(4,h), t);
		dot(two, (4,h), t);
		draw(two, (5,0)--(5,h), s);
		dot(two, (5,h), s);
		draw(two,(6,0)--(6,4*h), t);

		add(one); add(reflect((0,0),(1,0))*two);
		draw((-1,0)--(7,0));
		\end{asy}
		\caption{An example of composing two maps.}
	\end{figure}
\end{frame}


\end{document}
