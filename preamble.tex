%% If you need to define macros or include more packages, do all that here.
%% Otherwise leave this file alone.  DO NOT type your paper text in here.

\usepackage{amsmath,amsthm,amssymb}
\usepackage{enumerate}
\usepackage{hyperref}
\usepackage{asymptote}
\usepackage{graphicx}
\usepackage[english]{babel}

\usepackage{longtable}
\usepackage{paralist}

\usepackage{thmtools}
\usepackage{thm-restate}
% Uncomment the following lines to hide all figures
%\usepackage{comment}
%\excludecomment{figure}
%\let\endfigure\relax

% RSI Specific macros
\numberwithin{equation}{section}
\theoremstyle{definition}
\newtheorem{op}{Operation}
\newcommand{\dobarbell}[2]{
	\vcenter{\hbox{\includegraphics[scale=#1]{barbell/#2.pdf}}}
}
\newcommand{\barbell}[1]{
	\mathchoice%
	{\dobarbell{1.6}{#1}}
	{\dobarbell{1.2}{#1}}
	{\dobarbell{0.9}{#1}}
	{\dobarbell{0.9}{#1}}
}
\def\ul#1{\underline{#1}}

\def\stimes{\otimes_{R^s}}
\def\ttimes{\otimes_{R^t}}

\def\MM{\mathcal M}
\def\AA{\mathcal A}
\def\SS{\mathcal S}
\def\BB{\mathfrak B}
%\usepackage{dsfont}
%\def\HH{\mathds H}
\def\HH{\mathcal H}
\DeclareMathOperator{\Top}{Top}


% For code
\usepackage{listings}
\lstset{basicstyle=\ttfamily,
	numbers=left,
	numbersep=5pt,
	numberstyle=\tiny,
	keywordstyle=\bfseries,
	title=\lstname,
	showstringspaces=false,
	frame=single}
\usepackage{paralist}

\usepackage{tikz}
\usepackage{pgfplots}
%\pgfplotsset{
%    every axis/.append style={scale only axis,width=0.85\textwidth,},
%}


% Small commands
\newcommand{\myqed}{\textsc{Q.e.d.}}
\newcommand{\cbrt}[1]{\sqrt[3]{#1}}
\newcommand{\floor}[1]{\left\lfloor #1 \right\rfloor}
\newcommand{\ceiling}[1]{\left\lceil #1 \right\rceil}
\newcommand{\mailto}[1]{\href{mailto:#1}{#1}}
\renewcommand{\iff}{\Leftrightarrow}
\renewcommand{\implies}{\Rightarrow}
\newcommand{\hrulebar}{
  \par\hspace{\fill}\rule{0.95\linewidth}{.7pt}\hspace{\fill}
  \par\nointerlineskip \vspace{\baselineskip}
}
\def\half{\frac{1}{2}}

%For use in author command
\newcommand{\plusemail}[1]{\\ \normalfont \texttt{\mailto{#1}}}

%More commands and math operators
\DeclareMathOperator{\cis}{cis}
\DeclareMathOperator{\lcm}{lcm}

%Convenient Environments
\newenvironment{soln}{\begin{proof}[Solution]}{\end{proof}}
\newenvironment{parlist}{\begin{inparaenum}[(i)]}{\end{inparaenum}}
\newenvironment{gobble}{\setbox\z@\vbox\bgroup}{\egroup}

%Inequalities
\newcommand{\cycsum}{\sum_{\text{cyc}}}
\newcommand{\symsum}{\sum_{\text{sym}}}
\newcommand{\cycprod}{\prod_{\text{cyc}}}
\newcommand{\symprod}{\prod_{\text{sym}}}

%From H113 "Introduction to Abstract Algebra" at UC Berkeley
\def\CC{\mathbb C}
\def\FF{\mathbb F}
\def\NN{\mathbb N}
\def\QQ{\mathbb Q}
\def\RR{\mathbb R}
\def\ZZ{\mathbb Z}
\newcommand{\normal}{\trianglelefteq}
\newcommand{\charin}{\text{ char }}
\DeclareMathOperator{\sign}{sign}
\DeclareMathOperator{\Aut}{Aut}
\DeclareMathOperator{\Inn}{Inn}
\DeclareMathOperator{\Syl}{Syl}

%From Kiran Kedlaya's "Geometry Unbound"
\def\abs#1{\lvert #1 \rvert}
\def\norm#1{\lVert #1 \rVert}
\def\dang{\measuredangle} %% Directed angle
\def\line#1{\overleftrightarrow{#1}}
\def\ray#1{\overrightarrow{#1}} 
\def\seg#1{\overline{#1}}
\def\arc#1{\wideparen{#1}}

%From M275 "Topology" at SJSU
\newcommand{\id}{\text{id}}
\newcommand{\taking}[1]{\stackrel{#1}{\longrightarrow}}
\newcommand{\inv}{^{-1}}

%From M170 "Introduction to Graph Theory" at SJSU
\DeclareMathOperator{\diam}{diam}
\DeclareMathOperator{\ord}{ord}
\newcommand{\defeq}{\stackrel{\text{def}}{=}}

%From the USAMO .tex filse
\def\st{^{\text{st}}}
\def\nd{^{\text{nd}}}
\def\rd{^{\text{rd}}}
\def\th{^{\text{th}}}
\def\dg{^\circ}
\def\be{\begin{enumerate}}
\def\bee{\begin{enumerate} \ii}
\def\ee{\end{enumerate}}
\def\bi{\begin{itemize}}
\def\bii{\begin{itemize} \ii}
\def\ei{\end{itemize}}
\def\ii{\item}

%Asy commands
\begin{asydef}
	import olympiad;
	import cse5;
	pointpen = black;
	pathpen = black;
	pathfontpen = black;
	anglepen = black;
	anglefontpen = black;

	// pen s = blue, t = red + dashed + 0.6;
	pen s = blue, t = red;
	pen dot_s = blue, dot_t = red;
\end{asydef}

\theoremstyle{plain}
\newtheorem{theorem}{Theorem}[section]
\newtheorem{lemma}[theorem]{Lemma} 
\newtheorem{conjecture}[theorem]{Conjecture} 
\newtheorem{corollary}[theorem]{Corollary}
\newtheorem{proposition}[theorem]{Proposition}

%Starred versions
\newtheorem*{theorem*}{Theorem}
\newtheorem*{lemma*}{Lemma}
\newtheorem*{conjecture*}{Conjecture}
\newtheorem*{corollary*}{Corollary}
\newtheorem*{proposition*}{Proposition}

%Def-style theorems
\theoremstyle{definition}
\newtheorem{claim}[theorem]{Claim}
\newtheorem{definition}[theorem]{Definition}
\newtheorem{fact}[theorem]{Fact}

\newtheorem{algorithm}{Algorithm}
\newtheorem{answer}{Answer}
\newtheorem{case}{Case}
\newtheorem{note}{Note}
\newtheorem{remark}{Remark}
\newtheorem{ques}{Question}

%Examples, exercises, problems
\newtheorem{example}{Example}
\newtheorem{exercise}{Exercise}
\newtheorem{problem}{Problem}
\newtheorem{joke}{Joke}

\newtheorem*{algorithm*}{Algorithm}
\newtheorem*{answer*}{Answer}
\newtheorem*{case*}{Case}
\newtheorem*{claim*}{Claim}
\newtheorem*{definition*}{Definition}
\newtheorem*{example*}{Example}
\newtheorem*{exercise*}{Exercise}
\newtheorem*{fact*}{Fact}
\newtheorem*{note*}{Note}
\newtheorem*{joke*}{Joke}
\newtheorem*{problem*}{Problem}
\newtheorem*{ques*}{Question}
\newtheorem*{remark*}{Remark}

