%% If you need to define macros or include more packages, do all that here.
%% Otherwise leave this file alone.  DO NOT type your paper text in here.

\usepackage{amsmath,amsthm,amssymb}
\usepackage{enumerate}
\usepackage{hyperref}
\usepackage{asymptote}
\usepackage{graphicx}
\usepackage[english]{babel}

\usepackage{longtable}
\usepackage{paralist}
\usepackage{comment}

% Tricking papersubmit.
\begin{comment}
%%
%% This is file `aliasctr.sty',
%% generated with the docstrip utility.
%%
%% The original source files were:
%%
%% aliasctr.dtx  (with options: `code')
%% This is a generated file.
%% 
%% This file is part of the `thmtools' package.
%% The `thmtools' package has the LPPL maintenance status: maintained.
%% Current Maintainer is Ulrich M. Schwarz, ulmi@absatzen.de
%% 
%% Copyright (C) 2008-2012 by Ulrich M. Schwarz.
%% 
%% This file may be distributed and/or modified under the
%% conditions of the LaTeX Project Public License, version 1.3a.
%% This version is obtainable at
%% http://www.latex-project.org/lppl/lppl-1-3a.txt
%% 
%% 
\NeedsTeXFormat {LaTeX2e}
\ProvidesPackage {aliasctr}[2012/05/04 v63]
\def\aliasctr@f@llow#1#2\@nil#3{%
  \ifx#1\@elt
  \noexpand #3%
  \else
  \expandafter\aliasctr@f@llow#1\@elt\@nil{#1}%
  \fi
}
\newcommand\aliasctr@follow[1]{%
  \expandafter\aliasctr@f@llow
  \csname cl@#1\endcsname\@elt\@nil{\csname cl@#1\endcsname}%
}
\renewcommand*\@addtoreset[2]{\bgroup
   \edef\aliasctr@@truelist{\aliasctr@follow{#2}}%
  \let\@elt\relax
  \expandafter\@cons\aliasctr@@truelist{{#1}}%
\egroup}
\RequirePackage{remreset}
\renewcommand*\@removefromreset[2]{\bgroup
  \edef\aliasctr@@truelist{\aliasctr@follow{#2}}%
  \expandafter\let\csname c@#1\endcsname\@removefromreset
  \def\@elt##1{%
    \expandafter\ifx\csname c@##1\endcsname\@removefromreset
    \else
      \noexpand\@elt{##1}%
    \fi}%
  \expandafter\xdef\aliasctr@@truelist{%
    \aliasctr@@truelist}
\egroup}
\newcommand\@counteralias[2]{{%
    \def\@@gletover##1##2{%
      \expandafter\global
      \expandafter\let\csname ##1\expandafter\endcsname
      \csname ##2\endcsname
    }%
    \@ifundefined{c@#2}{\@nocounterr{#2}}{%
      \@ifdefinable{c@#1}{%
        \@@gletover{c@#1}{c@#2}%
        \@@gletover{the#1}{the#2}%
        \@@gletover{theH#1}{theH#2}%
        \@@gletover{p@#1}{p@#2}%
        \expandafter\global
        \expandafter\def\csname cl@#1\expandafter\endcsname
        \expandafter{\csname cl@#2\endcsname}%
        %\@addtoreset{#1}{@ckpt}%
      }%
    }%
}}
\endinput
%%
%% End of file `aliasctr.sty'.

%%
%% This is file `gettitlestring.sty',
%% generated with the docstrip utility.
%%
%% The original source files were:
%%
%% gettitlestring.dtx  (with options: `package')
%% 
%% This is a generated file.
%% 
%% Project: gettitlestring
%% Version: 2010/12/03 v1.4
%% 
%% Copyright (C) 2009, 2010 by
%%    Heiko Oberdiek <heiko.oberdiek at googlemail.com>
%% 
%% This work may be distributed and/or modified under the
%% conditions of the LaTeX Project Public License, either
%% version 1.3c of this license or (at your option) any later
%% version. This version of this license is in
%%    http://www.latex-project.org/lppl/lppl-1-3c.txt
%% and the latest version of this license is in
%%    http://www.latex-project.org/lppl.txt
%% and version 1.3 or later is part of all distributions of
%% LaTeX version 2005/12/01 or later.
%% 
%% This work has the LPPL maintenance status "maintained".
%% 
%% This Current Maintainer of this work is Heiko Oberdiek.
%% 
%% The Base Interpreter refers to any `TeX-Format',
%% because some files are installed in TDS:tex/generic//.
%% 
%% This work consists of the main source file gettitlestring.dtx
%% and the derived files
%%    gettitlestring.sty, gettitlestring.pdf, gettitlestring.ins,
%%    gettitlestring.drv, gettitlestring-test1.tex,
%%    gettitlestring-test2.tex.
%% 
\begingroup\catcode61\catcode48\catcode32=10\relax%
  \catcode13=5 % ^^M
  \endlinechar=13 %
  \catcode35=6 % #
  \catcode39=12 % '
  \catcode44=12 % ,
  \catcode45=12 % -
  \catcode46=12 % .
  \catcode58=12 % :
  \catcode64=11 % @
  \catcode123=1 % {
  \catcode125=2 % }
  \expandafter\let\expandafter\x\csname ver@gettitlestring.sty\endcsname
  \ifx\x\relax % plain-TeX, first loading
  \else
    \def\empty{}%
    \ifx\x\empty % LaTeX, first loading,
      % variable is initialized, but \ProvidesPackage not yet seen
    \else
      \expandafter\ifx\csname PackageInfo\endcsname\relax
        \def\x#1#2{%
          \immediate\write-1{Package #1 Info: #2.}%
        }%
      \else
        \def\x#1#2{\PackageInfo{#1}{#2, stopped}}%
      \fi
      \x{gettitlestring}{The package is already loaded}%
      \aftergroup\endinput
    \fi
  \fi
\endgroup%
\begingroup\catcode61\catcode48\catcode32=10\relax%
  \catcode13=5 % ^^M
  \endlinechar=13 %
  \catcode35=6 % #
  \catcode39=12 % '
  \catcode40=12 % (
  \catcode41=12 % )
  \catcode44=12 % ,
  \catcode45=12 % -
  \catcode46=12 % .
  \catcode47=12 % /
  \catcode58=12 % :
  \catcode64=11 % @
  \catcode91=12 % [
  \catcode93=12 % ]
  \catcode123=1 % {
  \catcode125=2 % }
  \expandafter\ifx\csname ProvidesPackage\endcsname\relax
    \def\x#1#2#3[#4]{\endgroup
      \immediate\write-1{Package: #3 #4}%
      \xdef#1{#4}%
    }%
  \else
    \def\x#1#2[#3]{\endgroup
      #2[{#3}]%
      \ifx#1\@undefined
        \xdef#1{#3}%
      \fi
      \ifx#1\relax
        \xdef#1{#3}%
      \fi
    }%
  \fi
\expandafter\x\csname ver@gettitlestring.sty\endcsname
\ProvidesPackage{gettitlestring}%
  [2010/12/03 v1.4 Cleanup title references (HO)]%
\begingroup\catcode61\catcode48\catcode32=10\relax%
  \catcode13=5 % ^^M
  \endlinechar=13 %
  \catcode123=1 % {
  \catcode125=2 % }
  \catcode64=11 % @
  \def\x{\endgroup
    \expandafter\edef\csname GTS@AtEnd\endcsname{%
      \endlinechar=\the\endlinechar\relax
      \catcode13=\the\catcode13\relax
      \catcode32=\the\catcode32\relax
      \catcode35=\the\catcode35\relax
      \catcode61=\the\catcode61\relax
      \catcode64=\the\catcode64\relax
      \catcode123=\the\catcode123\relax
      \catcode125=\the\catcode125\relax
    }%
  }%
\x\catcode61\catcode48\catcode32=10\relax%
\catcode13=5 % ^^M
\endlinechar=13 %
\catcode35=6 % #
\catcode64=11 % @
\catcode123=1 % {
\catcode125=2 % }
\def\TMP@EnsureCode#1#2{%
  \edef\GTS@AtEnd{%
    \GTS@AtEnd
    \catcode#1=\the\catcode#1\relax
  }%
  \catcode#1=#2\relax
}
\TMP@EnsureCode{42}{12}% *
\TMP@EnsureCode{44}{12}% ,
\TMP@EnsureCode{45}{12}% -
\TMP@EnsureCode{46}{12}% .
\TMP@EnsureCode{47}{12}% /
\TMP@EnsureCode{91}{12}% [
\TMP@EnsureCode{93}{12}% ]
\edef\GTS@AtEnd{\GTS@AtEnd\noexpand\endinput}
\RequirePackage{kvoptions}[2009/07/17]
\SetupKeyvalOptions{%
  family=gettitlestring,%
  prefix=GTS@%
}
\newcommand*{\GetTitleStringSetup}{%
  \setkeys{gettitlestring}%
}
\DeclareBoolOption{expand}
\InputIfFileExists{gettitlestring.cfg}{}{}
\ProcessKeyvalOptions*\relax
\newcommand*{\GetTitleString}{%
  \ifGTS@expand
    \expandafter\GetTitleStringExpand
  \else
    \expandafter\GetTitleStringNonExpand
  \fi
}
\newcommand{\GetTitleStringExpand}[1]{%
  \def\GetTitleStringResult{#1}%
  \begingroup
    \GTS@DisablePredefinedCmds
    \GTS@DisableHook
    \edef\x{\endgroup
      \noexpand\def\noexpand\GetTitleStringResult{%
        \GetTitleStringResult
      }%
    }%
  \x
}
\newcommand{\GetTitleStringNonExpand}[1]{%
  \def\GetTitleStringResult{#1}%
  \global\let\GTS@GlobalString\GetTitleStringResult
  \begingroup
    \GTS@RemoveLeft
    \GTS@RemoveRight
  \endgroup
  \let\GetTitleStringResult\GTS@GlobalString
}
\def\GTS@DisablePredefinedCmds{%
  \let\label\@gobble
  \let\zlabel\@gobble
  \let\zref@label\@gobble
  \let\zref@labelbylist\@gobbletwo
  \let\zref@labelbyprops\@gobbletwo
  \let\index\@gobble
  \let\glossary\@gobble
  \let\markboth\@gobbletwo
  \let\@mkboth\@gobbletwo
  \let\markright\@gobble
  \let\phantomsection\@empty
  \def\addcontentsline{\expandafter\@gobble\@gobbletwo}%
  \let\raggedright\@empty
  \let\raggedleft\@empty
  \let\centering\@empty
  \let\protect\@unexpandable@protect
  \let\enit@format\@empty % package enumitem
}
\providecommand*{\GTS@DisableHook}{}
\def\GetTitleStringDisableCommands{%
  \begingroup
    \makeatletter
    \GTS@DisableCommands
}
\long\def\GTS@DisableCommands#1{%
    \toks0=\expandafter{\GTS@DisableHook}%
    \toks2={#1}%
    \xdef\GTS@GlobalString{\the\toks0 \the\toks2}%
  \endgroup
  \let\GTS@DisableHook\GTS@GlobalString
}
\def\GTS@RemoveLeft{%
  \toks@\expandafter\expandafter\expandafter{%
    \expandafter\GTS@Car\GTS@GlobalString{}{}{}{}\GTS@Nil
  }%
  \edef\GTS@Token{\the\toks@}%
  \GTS@PredefinedLeftCmds
  \expandafter\futurelet\expandafter\GTS@Token
  \expandafter\GTS@TestLeftSpace\GTS@GlobalString\GTS@Nil
  \GTS@End
}
\def\GTS@End{}
\long\def\GTS@TestLeft#1#2{%
  \def\GTS@temp{#1}%
  \ifx\GTS@temp\GTS@Token
    \toks@\expandafter\expandafter\expandafter{%
      \expandafter#2\GTS@GlobalString\GTS@Nil
    }%
    \expandafter\GTS@TestLeftEnd
  \fi
}
\long\def\GTS@TestLeftEnd#1\GTS@End{%
  \xdef\GTS@GlobalString{\the\toks@}%
  \GTS@RemoveLeft
}
\long\def\GTS@Car#1#2\GTS@Nil{#1}
\long\def\GTS@Cdr#1#2\GTS@Nil{#2}
\long\def\GTS@CdrTwo#1#2#3\GTS@Nil{#3}
\long\def\GTS@CdrThree#1#2#3#4\GTS@Nil{#4}
\long\def\GTS@CdrFour#1#2#3#4#5\GTS@Nil{#5}
\long\def\GTS@TestLeftSpace#1\GTS@Nil{%
  \ifx\GTS@Token\@sptoken
    \toks@\expandafter{%
      \romannumeral-0\GTS@GlobalString
    }%
    \expandafter\GTS@TestLeftEnd
  \fi
}
\def\GTS@PredefinedLeftCmds{%
  \GTS@TestLeft\Hy@phantomsection\GTS@Cdr
  \GTS@TestLeft\Hy@SectionAnchor\GTS@Cdr
  \GTS@TestLeft\Hy@SectionAnchorHref\GTS@CdrTwo
  \GTS@TestLeft\label\GTS@CdrTwo
  \GTS@TestLeft\zlabel\GTS@CdrTwo
  \GTS@TestLeft\index\GTS@CdrTwo
  \GTS@TestLeft\glossary\GTS@CdrTwo
  \GTS@TestLeft\markboth\GTS@CdrThree
  \GTS@TestLeft\@mkboth\GTS@CdrThree
  \GTS@TestLeft\addcontentsline\GTS@CdrFour
  \GTS@TestLeft\enit@format\GTS@Cdr % package enumitem
}
\def\GTS@RemoveRight{%
  \toks@{}%
  \expandafter\GTS@TestRightLabel\GTS@GlobalString
      \label{}\GTS@Nil\@nil
  \GTS@RemoveRightSpace
}
\begingroup
  \def\GTS@temp#1{\endgroup
    \def\GTS@RemoveRightSpace{%
      \expandafter\GTS@TestRightSpace\GTS@GlobalString
          \GTS@Nil#1\GTS@Nil\@nil
    }%
  }%
\GTS@temp{ }
\def\GTS@TestRightSpace#1 \GTS@Nil#2\@nil{%
  \ifx\relax#2\relax
  \else
    \gdef\GTS@GlobalString{#1}%
    \expandafter\GTS@RemoveRightSpace
  \fi
}
\def\GTS@TestRightLabel#1\label#2#3\GTS@Nil#4\@nil{%
  \def\GTS@temp{#3}%
  \ifx\GTS@temp\@empty
    \expandafter\gdef\expandafter\GTS@GlobalString\expandafter{%
      \the\toks@
      #1%
    }%
    \expandafter\@gobble
  \else
    \expandafter\@firstofone
  \fi
  {%
    \toks@\expandafter{\the\toks@#1}%
    \GTS@TestRightLabel#3\GTS@Nil\@nil
  }%
}
\GTS@AtEnd%
\endinput
%%
%% End of file `gettitlestring.sty'.

%%
%% This is file `parseargs.sty',
%% generated with the docstrip utility.
%%
%% The original source files were:
%%
%% parseargs.dtx  (with options: `parseargs')
%% This is a generated file.
%% 
%% This file is part of the `thmtools' package.
%% The `thmtools' package has the LPPL maintenance status: maintained.
%% Current Maintainer is Ulrich M. Schwarz, ulmi@absatzen.de
%% 
%% Copyright (C) 2008-2012 by Ulrich M. Schwarz.
%% 
%% This file may be distributed and/or modified under the
%% conditions of the LaTeX Project Public License, version 1.3a.
%% This version is obtainable at
%% http://www.latex-project.org/lppl/lppl-1-3a.txt
%% 
%% 
\NeedsTeXFormat {LaTeX2e}
\ProvidesPackage {parseargs}[2012/05/04 v63]

\newtoks\@parsespec
\def\parse@endquark{\parse@endquark}
\newcommand\parse[1]{%
  \@parsespec{#1\parse@endquark}\@parse}

\newcommand\@parse{%
  \edef\p@tmp{\the\@parsespec}%
  \ifx\p@tmp\parse@endquark
    \expandafter\@gobble
  \else
    \expandafter\@firstofone
  \fi{%
    \@parsepop
  }%
}
\def\@parsepop{%
  \expandafter\p@rsepop\the\@parsespec\@nil
  \@parsecmd
}
\def\p@rsepop#1#2\@nil{%
  #1%
  \@parsespec{#2}%
}

\newcommand\parseOpt[4]{%
  %\parseOpt{openchar}{closechar}{yes}{no}
  \def\@parsecmd{%
    \@ifnextchar#1{\@@reallyparse}{#4\@parse}%
  }%
  \def\@@reallyparse#1##1#2{%
    #3\@parse
  }%
}

\newcommand\parseMand[1]{%
  %\parseMand{code}
  \def\@parsecmd##1{#1\@parse}%
}

\newcommand\parseFlag[3]{%
  %\parseFlag{flagchar}{yes}{no}
  \def\@parsecmd{%
    \@ifnextchar#1{#2\expandafter\@parse\@gobble}{#3\@parse}%
  }%
}
\endinput
%%
%% End of file `parseargs.sty'.

%%
%% This is file `thm-amsthm.sty',
%% generated with the docstrip utility.
%%
%% The original source files were:
%%
%% thm-amsthm.dtx  (with options: `amsthm')
%% This is a generated file.
%% 
%% This file is part of the `thmtools' package.
%% The `thmtools' package has the LPPL maintenance status: maintained.
%% Current Maintainer is Ulrich M. Schwarz, ulmi@absatzen.de
%% 
%% Copyright (C) 2008-2012 by Ulrich M. Schwarz.
%% 
%% This file may be distributed and/or modified under the
%% conditions of the LaTeX Project Public License, version 1.3a.
%% This version is obtainable at
%% http://www.latex-project.org/lppl/lppl-1-3a.txt
%% 
%% 
\NeedsTeXFormat {LaTeX2e}
\ProvidesPackage {thm-amsthm}[2012/05/04 v63]
\providecommand\thmt@space{ }

\define@key{thmstyle}{spaceabove}{%
  \def\thmt@style@spaceabove{#1}%
}
\define@key{thmstyle}{spacebelow}{%
  \def\thmt@style@spacebelow{#1}%
}
\define@key{thmstyle}{headfont}{%
  \def\thmt@style@headfont{#1}%
}
\define@key{thmstyle}{bodyfont}{%
  \def\thmt@style@bodyfont{#1}%
}
\define@key{thmstyle}{notefont}{%
  \def\thmt@style@notefont{#1}%
}
\define@key{thmstyle}{headpunct}{%
  \def\thmt@style@headpunct{#1}%
}
\define@key{thmstyle}{notebraces}{%
  \def\thmt@style@notebraces{\thmt@embrace#1}%
}
\define@key{thmstyle}{break}[]{%
  \def\thmt@style@postheadspace{\newline}%
}
\define@key{thmstyle}{postheadspace}{%
  \def\thmt@style@postheadspace{#1}%
}
\define@key{thmstyle}{headindent}{%
  \def\thmt@style@headindent{#1}%
}

\newtoks\thmt@style@headstyle
\define@key{thmstyle}{headformat}[]{%
  \thmt@setheadstyle{#1}%
}
\define@key{thmstyle}{headstyle}[]{%
  \thmt@setheadstyle{#1}%
}
\def\thmt@setheadstyle#1{%
  \thmt@style@headstyle{%
    \def\NAME{\the\thm@headfont ##1}%
    \def\NUMBER{\bgroup\@upn{##2}\egroup}%
    \def\NOTE{\if=##3=\else\bgroup\thmt@space\the\thm@notefont(##3)\egroup\fi}%
  }%
  \def\thmt@tmp{#1}%
  \@onelevel@sanitize\thmt@tmp
  %\tracingall
  \ifcsname thmt@headstyle@\thmt@tmp\endcsname
    \thmt@style@headstyle\@xa{%
      \the\thmt@style@headstyle
      \csname thmt@headstyle@#1\endcsname
    }%
  \else
    \thmt@style@headstyle\@xa{%
      \the\thmt@style@headstyle
      #1%
    }%
  \fi
  %\showthe\thmt@style@headstyle
}
\def\thmt@headstyle@margin{%
  \makebox[0pt][r]{\NUMBER\ }\NAME\NOTE
}
\def\thmt@headstyle@swapnumber{%
  \NUMBER\ \NAME\NOTE
}

\def\thmt@embrace#1#2(#3){#1#3#2}

\def\thmt@declaretheoremstyle@setup{%
  \let\thmt@style@notebraces\@empty%
  \thmt@style@headstyle{}%
  \kvsetkeys{thmstyle}{%
    spaceabove=3pt,
    spacebelow=3pt,
    headfont=\bfseries,
    bodyfont=\normalfont,
    headpunct={.},
    postheadspace={ },
    headindent={},
    notefont={\fontseries\mddefault\upshape}
  }%
}
\def\thmt@declaretheoremstyle#1{%
  %\show\thmt@style@spaceabove
  \thmt@toks{\newtheoremstyle{#1}}%
  \thmt@toks\@xa\@xa\@xa{\@xa\the\@xa\thmt@toks\@xa{\thmt@style@spaceabove}}%
  \thmt@toks\@xa\@xa\@xa{\@xa\the\@xa\thmt@toks\@xa{\thmt@style@spacebelow}}%
  \thmt@toks\@xa\@xa\@xa{\@xa\the\@xa\thmt@toks\@xa{\thmt@style@bodyfont}}%
  \thmt@toks\@xa\@xa\@xa{\@xa\the\@xa\thmt@toks\@xa{\thmt@style@headindent}}% indent1 FIXME
  \thmt@toks\@xa\@xa\@xa{\@xa\the\@xa\thmt@toks\@xa{\thmt@style@headfont}}%
  \thmt@toks\@xa\@xa\@xa{\@xa\the\@xa\thmt@toks\@xa{\thmt@style@headpunct}}%
  \thmt@toks\@xa\@xa\@xa{\@xa\the\@xa\thmt@toks\@xa{\thmt@style@postheadspace}}%
  \thmt@toks\@xa\@xa\@xa{\@xa\the\@xa\thmt@toks\@xa{\the\thmt@style@headstyle}}% headspec FIXME
  \the\thmt@toks
  %1 Indent amount: empty = no indent, \parindent = normal paragraph indent
  %2 Space after theorem head: { } = normal interword space; \newline = linebreak
  %% BUGFIX: amsthm ignores notefont setting altogether:
  \thmt@toks\@xa\@xa\@xa{\csname th@#1\endcsname}%
  \thmt@toks
  \@xa\@xa\@xa\@xa\@xa\@xa\@xa{%
  \@xa\@xa\@xa\@xa\@xa\@xa\@xa\thm@notefont
  \@xa\@xa\@xa\@xa\@xa\@xa\@xa{%
  \@xa\@xa\@xa\thmt@style@notefont
  \@xa\thmt@style@notebraces
  \@xa}\the\thmt@toks}%
  \@xa\def\csname th@#1\@xa\endcsname\@xa{\the\thmt@toks}%
}

\define@key{thmdef}{qed}[\qedsymbol]{%
  \thmt@trytwice{}{%
    \addtotheorempostheadhook[\thmt@envname]{%
      \protected@edef\qedsymbol{#1}%
      \pushQED{\qed}%
    }%
    \addtotheoremprefoothook[\thmt@envname]{%
      \protected@edef\qedsymbol{#1}%
      \popQED
    }%
  }%
}

\def\thmt@amsthmlistbreakhack{%
  \leavevmode
  \vspace{-\baselineskip}%
  \par
  \everypar{\setbox\z@\lastbox\everypar{}}%
}

\define@key{thmuse}{listhack}[\relax]{%
  \addtotheorempostheadhook[local]{%
    \thmt@amsthmlistbreakhack
  }%
}

\endinput
%%
%% End of file `thm-amsthm.sty'.

%%
%% This is file `thm-autoref.sty',
%% generated with the docstrip utility.
%%
%% The original source files were:
%%
%% thm-autoref.dtx  (with options: `autoref')
%% This is a generated file.
%% 
%% This file is part of the `thmtools' package.
%% The `thmtools' package has the LPPL maintenance status: maintained.
%% Current Maintainer is Ulrich M. Schwarz, ulmi@absatzen.de
%% 
%% Copyright (C) 2008-2012 by Ulrich M. Schwarz.
%% 
%% This file may be distributed and/or modified under the
%% conditions of the LaTeX Project Public License, version 1.3a.
%% This version is obtainable at
%% http://www.latex-project.org/lppl/lppl-1-3a.txt
%% 
%% 
\NeedsTeXFormat {LaTeX2e}
\ProvidesPackage {thm-autoref}[2012/05/04 v63]

\RequirePackage{thm-patch, aliasctr, parseargs, keyval}

\let\@xa=\expandafter
\let\@nx=\noexpand

\newcommand\thmt@autorefsetup{%
  \@xa\def\csname\thmt@envname autorefname\@xa\endcsname\@xa{\thmt@thmname}%
  \ifthmt@hassibling
    \@counteralias{\thmt@envname}{\thmt@sibling}%
    \@xa\def\@xa\thmt@autoreffix\@xa{%
      \@xa\let\csname the\thmt@envname\@xa\endcsname
        \csname the\thmt@sibling\endcsname
      \def\thmt@autoreffix{}%
    }%
    \protected@edef\thmt@sibling{\thmt@envname}%
  \fi
}
\g@addto@macro\thmt@newtheorem@predefinition{\thmt@autorefsetup}%
\g@addto@macro\thmt@newtheorem@postdefinition{\csname thmt@autoreffix\endcsname}%

\def\thmt@refnamewithcomma #1#2#3,#4,#5\@nil{%
  \@xa\def\csname\thmt@envname #1utorefname\endcsname{#3}%
  \ifcsname #2refname\endcsname
    \csname #2refname\endcsname{\thmt@envname}{#3}{#4}%
  \fi
}
\define@key{thmdef}{refname}{\thmt@trytwice{}{%
  \thmt@refnamewithcomma{a}{c}#1,\textbf{?? (pl. #1)},\@nil
}}
\define@key{thmdef}{Refname}{\thmt@trytwice{}{%
  \thmt@refnamewithcomma{A}{C}#1,\textbf{?? (pl. #1)},\@nil
}}

\ifcsname Autoref\endcsname\else
\let\thmt@HyRef@testreftype\HyRef@testreftype
\def\HyRef@Testreftype#1.#2\\{%
  \ltx@IfUndefined{#1Autorefname}{%
    \thmt@HyRef@testreftype#1.#2\\%
  }{%
    \edef\HyRef@currentHtag{%
      \expandafter\noexpand\csname#1Autorefname\endcsname
      \noexpand~%
    }%
  }%
}

\let\thmt@HyPsd@@autorefname\HyPsd@@autorefname
\def\HyPsd@@Autorefname#1.#2\@nil{%
  \tracingall
  \ltx@IfUndefined{#1Autorefname}{%
    \thmt@HyPsd@@autorefname#1.#2\@nil
  }{%
    \csname#1Autorefname\endcsname\space
  }%
}%
\def\Autoref{%
  \parse{%
  {\parseFlag*{\def\thmt@autorefstar{*}}{\let\thmt@autorefstar\@empty}}%
  {\parseMand{%
    \bgroup
    \let\HyRef@testreftype\HyRef@Testreftype
    \let\HyPsd@@autorefname\HyPsd@@Autorefname
    \@xa\autoref\thmt@autorefstar{##1}%
    \egroup
    \let\@parsecmd\@empty
  }}%
  }%
}
\fi % ifcsname Autoref

\AtBeginDocument{%
  \@ifpackageloaded{nameref}{%
    \addtotheorempostheadhook{%
      \expandafter\NR@gettitle\expandafter{\thmt@shortoptarg}%
  }}{}
}

\AtBeginDocument{%
  \@ifpackageloaded{cleveref}{%
    \@ifpackagelater{cleveref}{2010/04/30}{%
    % OK, new enough
    }{%
      \PackageWarningNoLine{thmtools}{%
        Your version of cleveref is too old!\MessageBreak
        Update to version 0.16.1 or later%
      }
    }
  }{}
}
\endinput
%%
%% End of file `thm-autoref.sty'.

%%
%% This is file `thm-beamer.sty',
%% generated with the docstrip utility.
%%
%% The original source files were:
%%
%% thm-beamer.dtx  (with options: `beamer')
%% This is a generated file.
%% 
%% This file is part of the `thmtools' package.
%% The `thmtools' package has the LPPL maintenance status: maintained.
%% Current Maintainer is Ulrich M. Schwarz, ulmi@absatzen.de
%% 
%% Copyright (C) 2008-2012 by Ulrich M. Schwarz.
%% 
%% This file may be distributed and/or modified under the
%% conditions of the LaTeX Project Public License, version 1.3a.
%% This version is obtainable at
%% http://www.latex-project.org/lppl/lppl-1-3a.txt
%% 
%% 
\NeedsTeXFormat {LaTeX2e}
\ProvidesPackage {thm-beamer}[2012/05/04 v63]
\newif\ifthmt@hasoverlay
\def\thmt@parsetheoremargs#1{%
  \parse{%
    {\parseOpt<>{\thmt@hasoverlaytrue\def\thmt@overlay{##1}}{}}%
    {\parseOpt[]{\def\thmt@optarg{##1}}{%
      \let\thmt@shortoptarg\@empty
      \let\thmt@optarg\@empty}}%
    {\ifthmt@hasoverlay\expandafter\@gobble\else\expandafter\@firstofone\fi
        {\parseOpt<>{\thmt@hasoverlaytrue\def\thmt@overlay{##1}}{}}%
    }%
    {%
      \def\thmt@local@preheadhook{}%
      \def\thmt@local@postheadhook{}%
      \def\thmt@local@prefoothook{}%
      \def\thmt@local@postfoothook{}%
      \thmt@local@preheadhook
      \csname thmt@#1@preheadhook\endcsname
      \thmt@generic@preheadhook
      \protected@edef\tmp@args{%
        \ifthmt@hasoverlay <\thmt@overlay>\fi
        \ifx\@empty\thmt@optarg\else [{\thmt@optarg}]\fi
      }%
      \csname thmt@original@#1\@xa\endcsname\tmp@args
      \thmt@local@postheadhook
      \csname thmt@#1@postheadhook\endcsname
      \thmt@generic@postheadhook
      \let\@parsecmd\@empty
    }%
  }
}%
\endinput
%%
%% End of file `thm-beamer.sty'.

%%
%% This is file `thmdef-mdframed.sty',
%% generated with the docstrip utility.
%%
%% The original source files were:
%%
%% thmdef-mdframed.dtx  (with options: `mdframed')
%% This is a generated file.
%% 
%% This file is part of the `thmtools' package.
%% The `thmtools' package has the LPPL maintenance status: maintained.
%% Current Maintainer is Ulrich M. Schwarz, ulmi@absatzen.de
%% 
%% Copyright (C) 2008-2012 by Ulrich M. Schwarz.
%% 
%% This file may be distributed and/or modified under the
%% conditions of the LaTeX Project Public License, version 1.3a.
%% This version is obtainable at
%% http://www.latex-project.org/lppl/lppl-1-3a.txt
%% 
%% 
\NeedsTeXFormat {LaTeX2e}
\ProvidesPackage {thmdef-mdframed}[2012/05/04 v63]
\define@key{thmdef}{mdframed}[{}]{%
  \thmt@trytwice{}{%
    \RequirePackage{mdframed}%
    \RequirePackage{thm-patch}%
    \addtotheorempreheadhook[\thmt@envname]{%
      \begin{mdframed}[#1]}%
    \addtotheorempostfoothook[\thmt@envname]{\end{mdframed}}%
    }%
}
\endinput
%%
%% End of file `thmdef-mdframed.sty'.

%%
%% This is file `thmdef-shaded.sty',
%% generated with the docstrip utility.
%%
%% The original source files were:
%%
%% thmdef-shaded.dtx  (with options: `shaded')
%% This is a generated file.
%% 
%% This file is part of the `thmtools' package.
%% The `thmtools' package has the LPPL maintenance status: maintained.
%% Current Maintainer is Ulrich M. Schwarz, ulmi@absatzen.de
%% 
%% Copyright (C) 2008-2012 by Ulrich M. Schwarz.
%% 
%% This file may be distributed and/or modified under the
%% conditions of the LaTeX Project Public License, version 1.3a.
%% This version is obtainable at
%% http://www.latex-project.org/lppl/lppl-1-3a.txt
%% 
%% 
\NeedsTeXFormat {LaTeX2e}
\ProvidesPackage {thmdef-shaded}[2012/05/04 v63]
  \define@key{thmdef}{shaded}[{}]{%
  \thmt@trytwice{}{%
    \RequirePackage{shadethm}%
    \RequirePackage{thm-patch}%
    \addtotheorempreheadhook[\thmt@envname]{%
      \setlength\shadedtextwidth{\linewidth}%
      \kvsetkeys{thmt@shade}{#1}\begin{shadebox}}%
    \addtotheorempostfoothook[\thmt@envname]{\end{shadebox}}%
    }%
  }
\define@key{thmt@shade}{textwidth}{\setlength\shadedtextwidth{#1}}
\define@key{thmt@shade}{bgcolor}{\thmt@definecolor{shadethmcolor}{#1}}
\define@key{thmt@shade}{rulecolor}{\thmt@definecolor{shaderulecolor}{#1}}
\define@key{thmt@shade}{rulewidth}{\setlength\shadeboxrule{#1}}
\define@key{thmt@shade}{margin}{\setlength\shadeboxsep{#1}}
\define@key{thmt@shade}{padding}{\setlength\shadeboxsep{#1}}
\define@key{thmt@shade}{leftmargin}{\setlength\shadeleftshift{#1}}
\define@key{thmt@shade}{rightmargin}{\setlength\shaderightshift{#1}}
\def\thmt@colorlet#1#2{%
  %\typeout{don't know how to let color `#1' be like color `#2'!}%
  \@xa\let\csname\string\color@#1\@xa\endcsname
    \csname\string\color@#2\endcsname
  % this is dubious at best, we don't know what a backend does.
}
\AtBeginDocument{%
  \ifcsname colorlet\endcsname
    \let\thmt@colorlet\colorlet
  \fi
}
\def\thmt@drop@relax#1\relax{}
\def\thmt@definecolor#1#2{%
  \thmt@def@color{#1}#2\thmt@drop@relax
    {gray}{0.5}%
    \thmt@colorlet{#1}{#2}%
  \relax
}
\def\thmt@def@color#1#2#{%
  \definecolor{#1}}
\endinput
%%
%% End of file `thmdef-shaded.sty'.

%%
%% This is file `thmdef-thmbox.sty',
%% generated with the docstrip utility.
%%
%% The original source files were:
%%
%% thmdef-thmbox.dtx  (with options: `thmbox')
%% This is a generated file.
%% 
%% This file is part of the `thmtools' package.
%% The `thmtools' package has the LPPL maintenance status: maintained.
%% Current Maintainer is Ulrich M. Schwarz, ulmi@absatzen.de
%% 
%% Copyright (C) 2008-2012 by Ulrich M. Schwarz.
%% 
%% This file may be distributed and/or modified under the
%% conditions of the LaTeX Project Public License, version 1.3a.
%% This version is obtainable at
%% http://www.latex-project.org/lppl/lppl-1-3a.txt
%% 
%% 
\NeedsTeXFormat {LaTeX2e}
\ProvidesPackage {thmdef-thmbox}[2012/05/04 v63]
\define@key{thmdef}{thmbox}[L]{%
  \thmt@trytwice{%
  \let\oldproof=\proof
  \let\oldendproof=\endproof
  \let\oldexample=\example
  \let\oldendexample=\endexample
  \RequirePackage[nothm]{thmbox}
  \let\proof=\oldproof
  \let\endproof=\oldendproof
  \let\example=\oldexample
  \let\endexample=\oldendexample
  \def\thmt@theoremdefiner{\newboxtheorem[#1]}%
  }{}%
}%
\endinput
%%
%% End of file `thmdef-thmbox.sty'.

%%
%% This is file `thm-kv.sty',
%% generated with the docstrip utility.
%%
%% The original source files were:
%%
%% thm-kv.dtx  (with options: `kv')
%% This is a generated file.
%% 
%% This file is part of the `thmtools' package.
%% The `thmtools' package has the LPPL maintenance status: maintained.
%% Current Maintainer is Ulrich M. Schwarz, ulmi@absatzen.de
%% 
%% Copyright (C) 2008-2012 by Ulrich M. Schwarz.
%% 
%% This file may be distributed and/or modified under the
%% conditions of the LaTeX Project Public License, version 1.3a.
%% This version is obtainable at
%% http://www.latex-project.org/lppl/lppl-1-3a.txt
%% 
%% 
\NeedsTeXFormat {LaTeX2e}
\ProvidesPackage {thm-kv}[2012/05/04 v63]

\let\@xa\expandafter
\let\@nx\noexpand

\DeclareOption{lowercase}{%
  \PackageInfo{thm-kv}{Theorem names will be lowercased}%
  \global\let\thmt@modifycase\MakeLowercase}

\DeclareOption{uppercase}{%
  \PackageInfo{thm-kv}{Theorem names will be uppercased}%
  \global\let\thmt@modifycase\MakeUppercase}

\DeclareOption{anycase}{%
  \PackageInfo{thm-kv}{Theorem names will be unchanged}%
  \global\let\thmt@modifycase\@empty}

\ExecuteOptions{uppercase}
\ProcessOptions\relax

\RequirePackage{keyval,kvsetkeys,thm-patch}

\long\def\thmt@kv@processor@default#1#2#3{%
 \def\kvsu@fam{#1}% new
 \@onelevel@sanitize\kvsu@fam% new
 \def\kvsu@key{#2}% new
 \@onelevel@sanitize\kvsu@key% new
 \unless\ifcsname KV@#1@\kvsu@key\endcsname
   \unless\ifcsname KVS@#1@handler\endcsname
     \kv@error@unknownkey{#1}{\kvsu@key}%
   \else
     \csname KVS@#1@handler\endcsname{#2}{#3}%
   % still using #2 #3 here is intentional: handler might
   % be used for strange stuff like implementing key names
   % that contain strange characters or other strange things.
     \relax
   \fi
 \else
   \ifx\kv@value\relax
     \unless\ifcsname KV@#1@\kvsu@key @default\endcsname
       \kv@error@novalue{#1}{\kvsu@key}%
     \else
       \csname KV@#1@\kvsu@key @default\endcsname
       \relax
     \fi
   \else
     \csname KV@#1@\kvsu@key\endcsname{#3}%
   \fi
 \fi
}

\@ifpackagelater{kvsetkeys}{2012/04/23}{%
  \PackageInfo{thm-kv}{kvsetkeys patch (v1.16 or later)}%
  \long\def\tmp@KVS@PD#1#2#3{%
    \def \kv@fam {#1}%
    \unless \ifcsname KV@#1@#2\endcsname
      \unless \ifcsname KVS@#1@handler\endcsname
        \kv@error@unknownkey {#1}{#2}%
      \else
        \kv@handled@true
        \csname KVS@#1@handler\endcsname {#2}{#3}\relax
        \ifkv@handled@ \else
          \kv@error@unknownkey {#1}{#2}%
        \fi
      \fi
    \else
      \ifx \kv@value \relax
        \unless \ifcsname KV@#1@#2@default\endcsname
          \kv@error@novalue {#1}{#2}%
        \else
          \csname KV@#1@#2@default\endcsname \relax
        \fi
      \else
        \csname KV@#1@#2\endcsname {#3}%
      \fi
    \fi
  }%
  \ifx\tmp@KVS@PD\KVS@ProcessorDefault
    \let\KVS@ProcessorDefault\thmt@kv@processor@default
    \def\kv@processor@default#1#2{%
      \begingroup
        \csname @safe@activestrue\endcsname
        \@xa\let\csname ifincsname\@xa\endcsname\csname iftrue\endcsname
        \edef\KVS@temp{\endgroup
          \noexpand\KVS@ProcessorDefault{#1}{\etex@unexpanded{#2}}%
        }%
        \KVS@temp
    }%
  \else
    \PackageError{thm-kv}{kvsetkeys patch failed}{Try kvsetkeys v1.16 or earlier}
  \fi
}{\@ifpackagelater{kvsetkeys}{2011/04/06}{%
  % Patch has disappeared somewhere... thanksalot.
  \PackageInfo{thm-kv}{kvsetkeys patch (v1.13 or later)}
  \long\def\tmp@KVS@PD#1#2#3{% no non-etex-support here...
    \unless\ifcsname KV@#1@#2\endcsname
     \unless\ifcsname KVS@#1@handler\endcsname
        \kv@error@unknownkey{#1}{#2}%
      \else
        \csname KVS@#1@handler\endcsname{#2}{#3}%
        \relax
      \fi
    \else
      \ifx\kv@value\relax
       \unless\ifcsname KV@#1@#2@default\endcsname
          \kv@error@novalue{#1}{#2}%
        \else
          \csname KV@#1@#2@default\endcsname
          \relax
        \fi
      \else
        \csname KV@#1@#2\endcsname{#3}%
      \fi
    \fi
  }%
  \ifx\tmp@KVS@PD\KVS@ProcessorDefault
    \let\KVS@ProcessorDefault\thmt@kv@processor@default
    \def\kv@processor@default#1#2{%
      \begingroup
        \csname @safe@activestrue\endcsname
        \let\ifincsname\iftrue
        \edef\KVS@temp{\endgroup
        \noexpand\KVS@ProcessorDefault{#1}{\unexpanded{#2}}%
      }%
    \KVS@temp
  }
  \else
    \PackageError{thm-kv}{kvsetkeys patch failed, try kvsetkeys v1.13 or earlier}
  \fi
}{%
  \RequirePackage{etex}
  \PackageInfo{thm-kv}{kvsetkeys patch applied (pre-1.13)}%
  \let\kv@processor@default\thmt@kv@processor@default
}}

\newcommand\thmt@mkignoringkeyhandler[1]{%
  \kv@set@family@handler{#1}{%
    \thmt@debug{Key `##1' with value `##2' ignored by #1.}%
  }%
}
\newcommand\thmt@mkextendingkeyhandler[3]{%
  \kv@set@family@handler{#1}{%
    \thmt@selfextendingkeyhandler{#1}{#2}{#3}%
      {##1}{##2}%
  }%
}

\newcommand\thmt@selfextendingkeyhandler[5]{%
  % #1: family
  % #2: prefix for file
  % #3: key hint for error
  % #4: actual key
  % #5: actual value
  \IfFileExists{#2-#4.sty}{%
    \PackageInfo{thmtools}%
      {Automatically pulling in `#2-#4'}%
    \RequirePackage{#2-#4}%
    \ifcsname KV@#1@#4\endcsname
      \csname KV@#1@#4\endcsname{#5}%
    \else
      \PackageError{thmtools}%
      {#3 `#4' not known}
      {I don't know what that key does.\MessageBreak
       I've even loaded the file `#2-#4.sty', but that didn't help.
      }%
    \fi
  }{%
    \PackageError{thmtools}%
    {#3 `#4' not known}
    {I don't know what that key does by myself,\MessageBreak
     and no file `#2-#4.sty' to tell me seems to exist.
    }%
  }%
}

\newif\if@thmt@firstkeyset

\def\thmt@trytwice{%
  \if@thmt@firstkeyset
    \@xa\@firstoftwo
  \else
    \@xa\@secondoftwo
  \fi
}

\@for\tmp@keyname:=parent,numberwithin,within\do{%
\define@key{thmdef}{\tmp@keyname}{\thmt@trytwice{\thmt@setparent{#1}}{}}%
}

\@for\tmp@keyname:=sibling,numberlike,sharenumber\do{%
\define@key{thmdef}{\tmp@keyname}{\thmt@trytwice{\thmt@setsibling{#1}}{}}%
}

\@for\tmp@keyname:=title,name,heading\do{%
\define@key{thmdef}{\tmp@keyname}{\thmt@trytwice{\thmt@setthmname{#1}}{}}%
}

\@for\tmp@keyname:=unnumbered,starred\do{%
\define@key{thmdef}{\tmp@keyname}[]{\thmt@trytwice{\thmt@isnumberedfalse}{}}%
}

\def\thmt@YES{yes}
\def\thmt@NO{no}
\def\thmt@UNIQUE{unless unique}
\define@key{thmdef}{numbered}[\thmt@YES]{
  \def\thmt@tmp{#1}%
  \thmt@trytwice{%
    \ifx\thmt@tmp\thmt@YES
      \thmt@isnumberedtrue
    \else\ifx\thmt@tmp\thmt@NO
      \thmt@isnumberedfalse
    \else\ifx\thmt@tmp\thmt@UNIQUE
      \RequirePackage[unq]{unique}
      \ifuniq{\thmt@envname}{%
        \thmt@isnumberedfalse
      }{%
        \thmt@isnumberedtrue
      }%
    \else
      \PackageError{thmtools}{Unknown value `#1' to key numbered}{}%
    \fi\fi\fi
  }{% trytwice: after definition
    \ifx\thmt@tmp\thmt@UNIQUE
      \addtotheorempreheadhook[\thmt@envname]{\setuniqmark{\thmt@envname}}%
      \addtotheorempreheadhook[\thmt@envname]{\def\thmt@dummyctrautorefname{\thmt@thmname\@gobble}}
    \fi
  }%
}

\define@key{thmdef}{preheadhook}{\thmt@trytwice{}{\addtotheorempreheadhook[\thmt@envname]{#1}}}
\define@key{thmdef}{postheadhook}{\thmt@trytwice{}{\addtotheorempostheadhook[\thmt@envname]{#1}}}
\define@key{thmdef}{prefoothook}{\thmt@trytwice{}{\addtotheoremprefoothook[\thmt@envname]{#1}}}
\define@key{thmdef}{postfoothook}{\thmt@trytwice{}{\addtotheorempostfoothook[\thmt@envname]{#1}}}

\define@key{thmdef}{style}{\thmt@trytwice{\thmt@setstyle{#1}}{}}

\define@key{thmdef0}{style}{%
  \ifcsname thmt@style #1@defaultkeys\endcsname
    \thmt@toks{\kvsetkeys{thmdef}}%
    \@xa\@xa\@xa\the\@xa\@xa\@xa\thmt@toks\@xa\@xa\@xa{%
      \csname thmt@style #1@defaultkeys\endcsname}%
  \fi
}
\thmt@mkignoringkeyhandler{thmdef0}

\def\thmt@setstyle#1{%
  \PackageWarning{thm-kv}{%
    Your backend doesn't have a `\string\theoremstyle' command.
  }%
}

\ifcsname theoremstyle\endcsname
  \let\thmt@originalthmstyle\theoremstyle
  \def\thmt@outerstyle{plain}
  \renewcommand\theoremstyle[1]{%
    \def\thmt@outerstyle{#1}%
    \thmt@originalthmstyle{#1}%
  }
  \def\thmt@setstyle#1{%
    \thmt@originalthmstyle{#1}%
  }
  \g@addto@macro\thmt@newtheorem@postdefinition{%
    \thmt@originalthmstyle{\thmt@outerstyle}%
  }
\fi

\newif\ifthmt@isnumbered
\newcommand\thmt@setparent[1]{%
  \def\thmt@parent{#1}%
}
\newcommand\thmt@setsibling{%
  \def\thmt@sibling
}
\newcommand\thmt@setthmname{%
  \def\thmt@thmname
}

\thmt@mkextendingkeyhandler{thmdef}{thmdef}{\string\declaretheorem\space key}

\let\thmt@newtheorem\newtheorem

\newcommand\declaretheorem[2][]{%
  % why was that here?
  %\let\thmt@theoremdefiner\thmt@original@newtheorem
  \def\thmt@envname{#2}%
  \thmt@setthmname{\thmt@modifycase #2}%
  \thmt@setparent{}%
  \thmt@setsibling{}%
  \thmt@isnumberedtrue%
  \@thmt@firstkeysettrue%
  \kvsetkeys{thmdef0}{#1}%
  \kvsetkeys{thmdef}{#1}%
  \protected@edef\thmt@tmp{%
    \@nx\thmt@newtheorem
    \ifthmt@isnumbered\else *\fi
    {#2}%
    \ifx\thmt@sibling\@empty\else [\thmt@sibling]\fi
    {\thmt@thmname}%
    \ifx\thmt@parent\@empty\else [\thmt@parent]\fi
    \relax% added so we can delimited-read everything later
    % (recall newtheorem is patched)
  }%\show\thmt@tmp
  \thmt@tmp
  % uniquely ugly kludge: some keys make only sense
  % afterwards.
  % and it gets kludgier: again, the default-inherited
  % keys need to have a go at it.
  \@thmt@firstkeysetfalse%
  \kvsetkeys{thmdef0}{#1}%
  \kvsetkeys{thmdef}{#1}%
}
\@onlypreamble\declaretheorem

\providecommand\thmt@quark{\thmt@quark}


\thmt@mkextendingkeyhandler{thmuse}{thmuse}{\thmt@envname\space optarg key}

\addtotheorempreheadhook{%
  \ifx\thmt@optarg\@empty\else
    \@xa\thmt@garbleoptarg\@xa{\thmt@optarg}\fi
}%

\newif\ifthmt@thmuse@iskv

\providecommand\thmt@garbleoptarg[1]{%
  \thmt@thmuse@iskvfalse
  \def\thmt@newoptarg{\@gobble}%
  \def\thmt@newoptargextra{}%
  \let\thmt@shortoptarg\@empty
  \def\thmt@warn@unusedkeys{}%
  \@for\thmt@fam:=\thmt@thmuse@families\do{%
    \kvsetkeys{\thmt@fam}{#1}%
  }%
  \ifthmt@thmuse@iskv
    \protected@edef\thmt@optarg{%
      \@xa\thmt@newoptarg
      \thmt@newoptargextra\@empty
    }%
    \ifx\thmt@shortoptarg\@empty
      \protected@edef\thmt@shortoptarg{\thmt@newoptarg\@empty}%
    \fi
    \thmt@warn@unusedkeys
  \else
    \def\thmt@optarg{#1}%
    \def\thmt@shortoptarg{#1}%
  \fi
}
\def\thmt@splitopt#1=#2\thmt@quark{%
  \def\thmt@tmpkey{#1}%
  \ifx\thmt@tmpkey\@empty
    \def\thmt@tmpkey{\thmt@quark}%
  \fi
  \@onelevel@sanitize\thmt@tmpkey
}

\def\thmt@thmuse@families{thm@track@keys}

\kv@set@family@handler{thm@track@keys}{%
  \@onelevel@sanitize\kv@key
  \@namedef{thmt@unusedkey@\kv@key}{%
    \PackageWarning{thmtools}{Unused key `#1'}%
  }%
  \@xa\g@addto@macro\@xa\thmt@warn@unusedkeys\@xa{%
    \csname thmt@unusedkey@\kv@key\endcsname
  }
}

\def\thmt@define@thmuse@key#1#2{%
  \g@addto@macro\thmt@thmuse@families{,#1}%
  \define@key{#1}{#1}{\thmt@thmuse@iskvtrue
    \@namedef{thmt@unusedkey@#1}{}%
    #2}%
  \thmt@mkignoringkeyhandler{#1}%
}

\thmt@define@thmuse@key{label}{%
  \addtotheorempostheadhook[local]{\label{#1}}%
}
\thmt@define@thmuse@key{name}{%
  \thmt@setnewoptarg #1\@iden%
}
\newcommand\thmt@setnewoptarg[1][]{%
  \def\thmt@shortoptarg{#1}\thmt@setnewlongoptarg
}
\def\thmt@setnewlongoptarg #1\@iden{%
  \def\thmt@newoptarg{#1\@iden}}

\providecommand\thmt@suspendcounter[2]{%
  \@xa\protected@edef\csname the#1\endcsname{#2}%
  \@xa\let\csname c@#1\endcsname\c@thmt@dummyctr
}

\providecommand\thmcontinues[1]{%
  \ifcsname hyperref\endcsname
    \hyperref[#1]{continuing}
  \else
    continuing
  \fi
  from p.\,\pageref{#1}%
}

\thmt@define@thmuse@key{continues}{%
  \thmt@suspendcounter{\thmt@envname}{\thmt@trivialref{#1}{??}}%
  \g@addto@macro\thmt@newoptarg{{, }%
    \thmcontinues{#1}%
    \@iden}%
}

\def\thmt@declaretheoremstyle@setup{}
\def\thmt@declaretheoremstyle#1{%
  \PackageWarning{thmtools}{Your backend doesn't allow styling theorems}{}
}
\newcommand\declaretheoremstyle[2][]{%
  \def\thmt@style{#2}%
  \@xa\def\csname thmt@style \thmt@style @defaultkeys\endcsname{}%
  \thmt@declaretheoremstyle@setup
  \kvsetkeys{thmstyle}{#1}%
  \thmt@declaretheoremstyle{#2}%
}
\@onlypreamble\declaretheoremstyle

\kv@set@family@handler{thmstyle}{%
  \@onelevel@sanitize\kv@value
  \@onelevel@sanitize\kv@key
  \PackageInfo{thmtools}{%
    Key `\kv@key' (with value `\kv@value')\MessageBreak
    is not a known style key.\MessageBreak
    Will pass this to every \string\declaretheorem\MessageBreak
    that uses `style=\thmt@style'%
  }%
  \ifx\kv@value\relax% no value given, don't pass on {}!
    \@xa\g@addto@macro\csname thmt@style \thmt@style @defaultkeys\endcsname{%
      #1,%
    }%
  \else
    \@xa\g@addto@macro\csname thmt@style \thmt@style @defaultkeys\endcsname{%
      #1={#2},%
    }%
  \fi
}
\endinput
%%
%% End of file `thm-kv.sty'.

%%
%% This is file `thm-listof.sty',
%% generated with the docstrip utility.
%%
%% The original source files were:
%%
%% thm-listof.dtx  (with options: `listof')
%% This is a generated file.
%% 
%% This file is part of the `thmtools' package.
%% The `thmtools' package has the LPPL maintenance status: maintained.
%% Current Maintainer is Ulrich M. Schwarz, ulmi@absatzen.de
%% 
%% Copyright (C) 2008-2012 by Ulrich M. Schwarz.
%% 
%% This file may be distributed and/or modified under the
%% conditions of the LaTeX Project Public License, version 1.3a.
%% This version is obtainable at
%% http://www.latex-project.org/lppl/lppl-1-3a.txt
%% 
%% 
\NeedsTeXFormat {LaTeX2e}
\ProvidesPackage {thm-listof}[2012/05/04 v63]
\let\@xa=\expandafter
\let\@nx=\noexpand
\RequirePackage{thm-patch,keyval,kvsetkeys}

\def\thmtlo@oldchapter{0}%
\newcommand\thmtlo@chaptervspacehack{}
\ifcsname c@chapter\endcsname
  \ifx\c@chapter\relax\else
    \def\thmtlo@chaptervspacehack{%
      \ifnum \value{chapter}=\thmtlo@oldchapter\relax\else
        % new chapter, add vspace to loe.
        \addtocontents{loe}{\protect\addvspace{10\p@}}%
        \xdef\thmtlo@oldchapter{\arabic{chapter}}%
      \fi
    }%
  \fi
\fi

\providecommand\listtheoremname{List of Theorems}
\newcommand\listoftheorems[1][]{%
  %% much hacking here to pick up the definition from the class
  %% without oodles of conditionals.
  \bgroup
  \setlisttheoremstyle{#1}%
  \let\listfigurename\listtheoremname
  \def\contentsline##1{%
    \csname thmt@contentsline@##1\endcsname{##1}%
  }%
  \@for\thmt@envname:=\thmt@allenvs\do{%
  \@xa\protected@edef\csname l@\thmt@envname\endcsname{% CHECK: why p@edef?
    \@nx\@dottedtocline{1}{1.5em}{\@nx\thmt@listnumwidth}%
  }%
  }%
  \let\thref@starttoc\@starttoc
  \def\@starttoc##1{\thref@starttoc{loe}}%
  % new hack: to allow multiple calls, we defer the opening of the
  % loe file to AtEndDocument time. This is before the aux file is
  % read back again, that is early enough.
  % TODO: is it? crosscheck include/includeonly!
  \@fileswfalse
  \AtEndDocument{%
    \if@filesw
      \@ifundefined{tf@loe}{%
        \expandafter\newwrite\csname tf@loe\endcsname
        \immediate\openout \csname tf@loe\endcsname \jobname.loe\relax
      }{}%
    \fi
  }%
  %\expandafter
  \listoffigures
  \egroup
}

\newcommand\setlisttheoremstyle[1]{%
  \kvsetkeys{thmt-listof}{#1}%
}
\define@key{thmt-listof}{numwidth}{\def\thmt@listnumwidth{#1}}
\define@key{thmt-listof}{ignore}[\thmt@allenvs]{\ignoretheorems{#1}}
\define@key{thmt-listof}{onlynamed}[\thmt@allenvs]{\onlynamedtheorems{#1}}
\define@key{thmt-listof}{show}[\thmt@allenvs]{\showtheorems{#1}}
\define@key{thmt-listof}{ignoreall}[true]{\ignoretheorems{\thmt@allenvs}}
\define@key{thmt-listof}{showall}[true]{\showtheorems{\thmt@allenvs}}

\providecommand\thmt@listnumwidth{2.3em}

\providecommand\thmtformatoptarg[1]{ (#1)}

\newcommand\thmt@mklistcmd{%
  \@xa\protected@edef\csname l@\thmt@envname\endcsname{% CHECK: why p@edef?
    \@nx\@dottedtocline{1}{1.5em}{\@nx\thmt@listnumwidth}%
  }%
  \ifthmt@isstarred
    \@xa\def\csname ll@\thmt@envname\endcsname{%
      \protect\numberline{\protect\let\protect\autodot\protect\@empty}%
      \thmt@thmname
      \ifx\@empty\thmt@shortoptarg\else\protect\thmtformatoptarg{\thmt@shortoptarg}\fi
    }%
  \else
    \@xa\def\csname ll@\thmt@envname\endcsname{%
      \protect\numberline{\csname the\thmt@envname\endcsname}%
      \thmt@thmname
      \ifx\@empty\thmt@shortoptarg\else\protect\thmtformatoptarg{\thmt@shortoptarg}\fi
    }%
  \fi
  \@xa\gdef\csname thmt@contentsline@\thmt@envname\endcsname{%
    \thmt@contentslineShow% default:show
  }%
}
\def\thmt@allenvs{\@gobble}
\newcommand\thmt@recordenvname{%
  \edef\thmt@allenvs{\thmt@allenvs,\thmt@envname}%
}
\g@addto@macro\thmt@newtheorem@predefinition{%
  \thmt@mklistcmd
  \thmt@recordenvname
}

\addtotheorempostheadhook{%
  \thmtlo@chaptervspacehack
  \addcontentsline{loe}{\thmt@envname}{%
    \csname ll@\thmt@envname\endcsname
  }%
}

\newcommand\showtheorems[1]{%
  \@for\thmt@thm:=#1\do{%
    \typeout{showing \thmt@thm}%
    \@xa\let\csname thmt@contentsline@\thmt@thm\endcsname
      =\thmt@contentslineShow
  }%
}

\newcommand\ignoretheorems[1]{%
  \@for\thmt@thm:=#1\do{%
    \@xa\let\csname thmt@contentsline@\thmt@thm\endcsname
      =\thmt@contentslineIgnore
  }%
}
\newcommand\onlynamedtheorems[1]{%
  \@for\thmt@thm:=#1\do{%
    \global\@xa\let\csname thmt@contentsline@\thmt@thm\endcsname
      =\thmt@contentslineIfNamed
  }%
}

\AtBeginDocument{%
\@ifpackageloaded{hyperref}{%
  \let\thmt@hygobble\@gobble
}{%
  \let\thmt@hygobble\@empty
}
\let\thmt@contentsline\contentsline
}

\def\thmt@contentslineIgnore#1#2#3{%
  \thmt@hygobble
}
\def\thmt@contentslineShow{%
  \thmt@contentsline
}

\def\thmt@contentslineIfNamed#1#2#3{%
  \thmt@ifhasoptname #2\thmtformatoptarg\@nil{%
    \thmt@contentslineShow{#1}{#2}{#3}%
  }{%
    \thmt@contentslineIgnore{#1}{#2}{#3}%
    %\thmt@contentsline{#1}{#2}{#3}%
  }
}

\def\thmt@ifhasoptname #1\thmtformatoptarg#2\@nil{%
  \ifx\@nil#2\@nil
    \@xa\@secondoftwo
  \else
    \@xa\@firstoftwo
  \fi
}
\endinput
%%
%% End of file `thm-listof.sty'.

%%
%% This is file `thm-llncs.sty',
%% generated with the docstrip utility.
%%
%% The original source files were:
%%
%% thm-llncs.dtx  (with options: `llncs')
%% This is a generated file.
%% 
%% This file is part of the `thmtools' package.
%% The `thmtools' package has the LPPL maintenance status: maintained.
%% Current Maintainer is Ulrich M. Schwarz, ulmi@absatzen.de
%% 
%% Copyright (C) 2008-2012 by Ulrich M. Schwarz.
%% 
%% This file may be distributed and/or modified under the
%% conditions of the LaTeX Project Public License, version 1.3a.
%% This version is obtainable at
%% http://www.latex-project.org/lppl/lppl-1-3a.txt
%% 
%% 
\NeedsTeXFormat {LaTeX2e}
\ProvidesPackage {thm-llncs}[2012/05/04 v63]
\@ifclasslater{llncs}{2010/04/15}{}{%
  \PackageWarningNoLine{thmtools}{%
    LLNCS.cls too old, not supported by thmtools
  }%
  \endinput}
\ifx\thmt@modifycase\@empty\else
  \PackageWarningNoLine{thmtools}{%
    LLNCS support disables automatic casing of theorem names
  }%
  \let\thmt@modifycase\@empty
\fi
%%
\providecommand\thmt@style@headfont{\normalfont\bfseries}
\providecommand\thmt@style@bodyfont{\normalfont\itshape}

\let\thmt@original@spnewtheorem\spnewtheorem
\let\thmt@theoremdefiner\thmt@original@spnewtheorem

\def\spnewtheorem{%
  \thmt@isstarredfalse
  \thmt@hassiblingfalse
  \thmt@hasparentfalse
  \parse{%
    {\parseFlag*{\thmt@isstarredtrue}{}}%
    {\parseMand{\def\thmt@envname{##1}}}%
    {\parseOpt[]{\thmt@hassiblingtrue\def\thmt@sibling{##1}}{}}%
    {\parseMand{%
      \def\thmt@thmname{##1}%
    }}%
    {\parseOpt[]{\thmt@hasparenttrue\def\thmt@parent{##1}}{}}%
    {\parseMand{\def\thmt@style@headfont{##1}}}%
    {\parseMand{\def\thmt@style@bodyfont{##1}}}%
    {\let\@parsecmd\thmt@spnewtheoremiv}%
  }%
}

\newcommand\thmt@spnewtheoremiv{%
  \thmt@newtheorem@predefinition
  % whee, now reassemble the whole shebang.
  \protected@edef\thmt@args{%
    \@nx\thmt@theoremdefiner%
    \ifthmt@isstarred *\fi
    {\thmt@envname}%
    \ifthmt@hassibling [\thmt@sibling]\fi
    {\thmt@thmname}%
    \ifthmt@hasparent [\thmt@parent]\fi
    {\thmt@style@headfont}{\thmt@style@bodyfont}%
  }
  \thmt@args
  \thmt@newtheorem@postdefinition
}

\define@key{thmdef}{headfont}{%
  \def\thmt@style@headfont{#1}%
}
\define@key{thmdef}{bodyfont}{%
\def\thmt@style@bodyfont{#1}%
}

\def\thmt@almost@spnewtheorem#1\relax{%
  \def\thm@tmpa{\spnewtheorem#1}%
  \@xa\@xa\@xa\thm@tmpa
    \@xa\@xa\@xa{\@xa\thmt@style@headfont\@xa}%
    \@xa{\thmt@style@bodyfont}%
}
\let\thmt@newtheorem\thmt@almost@spnewtheorem
\endinput
%%
%% End of file `thm-llncs.sty'.

%%
%% This is file `thm-ntheorem.sty',
%% generated with the docstrip utility.
%%
%% The original source files were:
%%
%% thm-ntheorem.dtx  (with options: `ntheorem')
%% This is a generated file.
%% 
%% This file is part of the `thmtools' package.
%% The `thmtools' package has the LPPL maintenance status: maintained.
%% Current Maintainer is Ulrich M. Schwarz, ulmi@absatzen.de
%% 
%% Copyright (C) 2008-2012 by Ulrich M. Schwarz.
%% 
%% This file may be distributed and/or modified under the
%% conditions of the LaTeX Project Public License, version 1.3a.
%% This version is obtainable at
%% http://www.latex-project.org/lppl/lppl-1-3a.txt
%% 
%% 
\NeedsTeXFormat {LaTeX2e}
\ProvidesPackage {thm-ntheorem}[2012/05/04 v63]

\providecommand\thmt@space{ }

\def\thmt@declaretheoremstyle@setup{}
\def\thmt@declaretheoremstyle#1{%
  \ifcsname th@#1\endcsname\else
    \@xa\let\csname th@#1\endcsname\th@plain
  \fi
}

\def\thmt@notsupported#1#2{%
  \PackageWarning{thmtools}{Key `#2' not supported by #1}{}%
}

\define@key{thmstyle}{spaceabove}{%
  \setlength\theorempreskipamount{#1}%
}
\define@key{thmstyle}{spacebelow}{%
  \setlength\theorempostskipamount{#1}%
}
\define@key{thmstyle}{headfont}{%
  \theoremheaderfont{#1}%
}
\define@key{thmstyle}{bodyfont}{%
  \theorembodyfont{#1}%
}
\define@key{thmstyle}{notefont}{%
  \thmt@notsupported{ntheorem}{notefont}%
}
\define@key{thmstyle}{headpunct}{%
  \theoremseparator{#1}%
}
\define@key{thmstyle}{notebraces}{%
  \thmt@notsupported{ntheorem}{notebraces}%
}
\define@key{thmstyle}{break}{%
  \theoremstyle{break}%
}
\define@key{thmstyle}{postheadspace}{%
  %\def\thmt@style@postheadspace{#1}%
  \@xa\g@addto@macro\csname thmt@style \thmt@style @defaultkeys\endcsname{%
      postheadhook={\hspace{-\labelsep}\hspace*{#1}},%
  }%
}

\define@key{thmstyle}{headindent}{%
  \thmt@notsupported{ntheorem}{headindent}%
}
\define@key{thmstyle}{qed}[\qedsymbol]{%
  \@ifpackagewith{ntheorem}{thmmarks}{%
    \theoremsymbol{#1}%
  }{%
    \thmt@notsupported
      {ntheorem without thmmarks option}%
      {headindent}%
  }%
}

\let\@upn=\textup
\define@key{thmstyle}{headformat}[]{%
  \def\thmt@tmp{#1}%
  \@onelevel@sanitize\thmt@tmp
  %\tracingall
  \ifcsname thmt@headstyle@\thmt@tmp\endcsname
    \newtheoremstyle{\thmt@style}{%
      \item[\hskip\labelsep\theorem@headerfont%
        \def\NAME{\theorem@headerfont ####1}%
        \def\NUMBER{\bgroup\@upn{####2}\egroup}%
        \def\NOTE{}%
        \csname thmt@headstyle@#1\endcsname
        \theorem@separator
      ]
    }{%
      \item[\hskip\labelsep\theorem@headerfont%
        \def\NAME{\theorem@headerfont ####1}%
        \def\NUMBER{\bgroup\@upn{####2}\egroup}%
        \def\NOTE{\if=####3=\else\bgroup\thmt@space(####3)\egroup\fi}%
        \csname thmt@headstyle@#1\endcsname
        \theorem@separator
      ]
    }
  \else
    \newtheoremstyle{\thmt@style}{%
      \item[\hskip\labelsep\theorem@headerfont%
        \def\NAME{\the\thm@headfont ####1}%
        \def\NUMBER{\bgroup\@upn{####2}\egroup}%
        \def\NOTE{}%
        #1%
        \theorem@separator
      ]
    }{%
      \item[\hskip\labelsep\theorem@headerfont%
        \def\NAME{\the\thm@headfont ####1}%
        \def\NUMBER{\bgroup\@upn{####2}\egroup}%
        \def\NOTE{\if=####3=\else\bgroup\thmt@space(####3)\egroup\fi}%
        #1%
        \theorem@separator
      ]
    }
  \fi
}

\def\thmt@headstyle@margin{%
  \makebox[0pt][r]{\NUMBER\ }\NAME\NOTE
}
\def\thmt@headstyle@swapnumber{%
  \NUMBER\ \NAME\NOTE
}

\endinput
%%
%% End of file `thm-ntheorem.sty'.

%%
%% This is file `thm-patch.sty',
%% generated with the docstrip utility.
%%
%% The original source files were:
%%
%% thm-patch.dtx  (with options: `patch')
%% This is a generated file.
%% 
%% This file is part of the `thmtools' package.
%% The `thmtools' package has the LPPL maintenance status: maintained.
%% Current Maintainer is Ulrich M. Schwarz, ulmi@absatzen.de
%% 
%% Copyright (C) 2008-2012 by Ulrich M. Schwarz.
%% 
%% This file may be distributed and/or modified under the
%% conditions of the LaTeX Project Public License, version 1.3a.
%% This version is obtainable at
%% http://www.latex-project.org/lppl/lppl-1-3a.txt
%% 
%% 
\NeedsTeXFormat {LaTeX2e}
\ProvidesPackage {thm-patch}[2012/05/04 v63]
\RequirePackage{parseargs}

\newif\ifthmt@isstarred
\newif\ifthmt@hassibling
\newif\ifthmt@hasparent

\def\thmt@parsetheoremargs#1{%
  \parse{%
    {\parseOpt[]{\def\thmt@optarg{##1}}{%
      \let\thmt@shortoptarg\@empty
      \let\thmt@optarg\@empty}}%
    {%
      \def\thmt@local@preheadhook{}%
      \def\thmt@local@postheadhook{}%
      \def\thmt@local@prefoothook{}%
      \def\thmt@local@postfoothook{}%
      \thmt@local@preheadhook
      \csname thmt@#1@preheadhook\endcsname
      \thmt@generic@preheadhook
      % change following to \@xa-orgy at some point?
      % forex, might have keyvals involving commands.
      %\protected@edef\tmp@args{%
      %  \ifx\@empty\thmt@optarg\else [{\thmt@optarg}]\fi
      %}%
      \ifx\@empty\thmt@optarg
        \def\tmp@args{}%
      \else
        \@xa\def\@xa\tmp@args\@xa{\@xa[\@xa{\thmt@optarg}]}%
      \fi
      \csname thmt@original@#1\@xa\endcsname\tmp@args
      %%moved down: \thmt@local@postheadhook
      %% (give postheadhooks a chance to re-set nameref data)
      \csname thmt@#1@postheadhook\endcsname
      \thmt@generic@postheadhook
      \thmt@local@postheadhook
      \let\@parsecmd\@empty
    }%
  }%
}%

\let\thmt@original@newtheorem\newtheorem
\let\thmt@theoremdefiner\thmt@original@newtheorem

\def\newtheorem{%
  \thmt@isstarredfalse
  \thmt@hassiblingfalse
  \thmt@hasparentfalse
  \parse{%
    {\parseFlag*{\thmt@isstarredtrue}{}}%
    {\parseMand{\def\thmt@envname{##1}}}%
    {\parseOpt[]{\thmt@hassiblingtrue\def\thmt@sibling{##1}}{}}%
    {\parseMand{\def\thmt@thmname{##1}}}%
    {\parseOpt[]{\thmt@hasparenttrue\def\thmt@parent{##1}}{}}%
    {\let\@parsecmd\thmt@newtheoremiv}%
  }%
}

\newcommand\thmt@newtheoremiv{%
  \thmt@newtheorem@predefinition
  % whee, now reassemble the whole shebang.
  \protected@edef\thmt@args{%
    \@nx\thmt@theoremdefiner%
    \ifthmt@isstarred *\fi
    {\thmt@envname}%
    \ifthmt@hassibling [\thmt@sibling]\fi
    {\thmt@thmname}%
    \ifthmt@hasparent [\thmt@parent]\fi
  }
  \thmt@args
  \thmt@newtheorem@postdefinition
}

\newcommand\thmt@newtheorem@predefinition{}
\newcommand\thmt@newtheorem@postdefinition{%
  \let\thmt@theoremdefiner\thmt@original@newtheorem
}

\g@addto@macro\thmt@newtheorem@predefinition{%
  \@xa\thmt@providetheoremhooks\@xa{\thmt@envname}%
}
\g@addto@macro\thmt@newtheorem@postdefinition{%
  \@xa\thmt@addtheoremhook\@xa{\thmt@envname}%
  \ifthmt@isstarred\@namedef{the\thmt@envname}{}\fi
  \protected@edef\thmt@tmp{%
    \def\@nx\thmt@envname{\thmt@envname}%
    \def\@nx\thmt@thmname{\thmt@thmname}%
  }%
  \@xa\addtotheorempreheadhook\@xa[\@xa\thmt@envname\@xa]\@xa{%
    \thmt@tmp
  }%
}
\newcommand\thmt@providetheoremhooks[1]{%
  \@namedef{thmt@#1@preheadhook}{}%
  \@namedef{thmt@#1@postheadhook}{}%
  \@namedef{thmt@#1@prefoothook}{}%
  \@namedef{thmt@#1@postfoothook}{}%
  \def\thmt@local@preheadhook{}%
  \def\thmt@local@postheadhook{}%
  \def\thmt@local@prefoothook{}%
  \def\thmt@local@postfoothook{}%
}
\newcommand\thmt@addtheoremhook[1]{%
  % this adds two command calls to the newly-defined theorem.
  \@xa\let\csname thmt@original@#1\@xa\endcsname
          \csname#1\endcsname
  \@xa\renewcommand\csname #1\endcsname{%
    \thmt@parsetheoremargs{#1}%
  }%
  \@xa\let\csname thmt@original@end#1\@xa\endcsname\csname end#1\endcsname
  \@xa\def\csname end#1\endcsname{%
    % these need to be in opposite order of headhooks.
    \csname thmtgeneric@prefoothook\endcsname
    \csname thmt@#1@prefoothook\endcsname
    \csname thmt@local@prefoothook\endcsname
    \csname thmt@original@end#1\endcsname
    \csname thmt@generic@postfoothook\endcsname
    \csname thmt@#1@postfoothook\endcsname
    \csname thmt@local@postfoothook\endcsname
  }%
}
\newcommand\thmt@generic@preheadhook{\refstepcounter{thmt@dummyctr}}
\newcommand\thmt@generic@postheadhook{}
\newcommand\thmt@generic@prefoothook{}
\newcommand\thmt@generic@postfoothook{}

\def\thmt@local@preheadhook{}
\def\thmt@local@postheadhook{}
\def\thmt@local@prefoothook{}
\def\thmt@local@postfoothook{}

\providecommand\g@prependto@macro[2]{%
  \begingroup
    \toks@\@xa{\@xa{#1}{#2}}%
    \def\tmp@a##1##2{##2##1}%
    \@xa\@xa\@xa\gdef\@xa\@xa\@xa#1\@xa\@xa\@xa{\@xa\tmp@a\the\toks@}%
  \endgroup
}

\newcommand\addtotheorempreheadhook[1][generic]{%
  \expandafter\g@addto@macro\csname thmt@#1@preheadhook\endcsname%
}
\newcommand\addtotheorempostheadhook[1][generic]{%
  \expandafter\g@addto@macro\csname thmt@#1@postheadhook\endcsname%
}

\newcommand\addtotheoremprefoothook[1][generic]{%
  \expandafter\g@prependto@macro\csname thmt@#1@prefoothook\endcsname%
}
\newcommand\addtotheorempostfoothook[1][generic]{%
  \expandafter\g@prependto@macro\csname thmt@#1@postfoothook\endcsname%
}

\ifx\proof\endproof\else% yup, that's a quaint way of doing it :)
  % FIXME: this assumes proof has the syntax of theorems, which
  % usually happens to be true (optarg overrides "Proof" string).
  % FIXME: refactor into thmt@addtheoremhook, but we really don't want to
  % call the generic-hook...
  \let\thmt@original@proof=\proof
  \renewcommand\proof{%
    \thmt@parseproofargs%
  }%
  \def\thmt@parseproofargs{%
    \parse{%
      {\parseOpt[]{\def\thmt@optarg{##1}}{\let\thmt@optarg\@empty}}%
      {%
        \thmt@proof@preheadhook
        %\thmt@generic@preheadhook
        \protected@edef\tmp@args{%
          \ifx\@empty\thmt@optarg\else [\thmt@optarg]\fi
        }%
        \csname thmt@original@proof\@xa\endcsname\tmp@args
        \thmt@proof@postheadhook
        %\thmt@generic@postheadhook
        \let\@parsecmd\@empty
      }%
    }%
  }%

  \let\thmt@original@endproof=\endproof
  \def\endproof{%
    % these need to be in opposite order of headhooks.
    %\csname thmtgeneric@prefoothook\endcsname
    \thmt@proof@prefoothook
    \thmt@original@endproof
    %\csname thmt@generic@postfoothook\endcsname
    \thmt@proof@postfoothook
  }%
  \@namedef{thmt@proof@preheadhook}{}%
  \@namedef{thmt@proof@postheadhook}{}%
  \@namedef{thmt@proof@prefoothook}{}%
  \@namedef{thmt@proof@postfoothook}{}%
\fi
\endinput
%%
%% End of file `thm-patch.sty'.

%%
%% This is file `thm-restate.sty',
%% generated with the docstrip utility.
%%
%% The original source files were:
%%
%% thm-restate.dtx  (with options: `restate')
%% This is a generated file.
%% 
%% This file is part of the `thmtools' package.
%% The `thmtools' package has the LPPL maintenance status: maintained.
%% Current Maintainer is Ulrich M. Schwarz, ulmi@absatzen.de
%% 
%% Copyright (C) 2008-2012 by Ulrich M. Schwarz.
%% 
%% This file may be distributed and/or modified under the
%% conditions of the LaTeX Project Public License, version 1.3a.
%% This version is obtainable at
%% http://www.latex-project.org/lppl/lppl-1-3a.txt
%% 
%% 
\NeedsTeXFormat {LaTeX2e}
\ProvidesPackage {thm-restate}[2012/05/04 v63]
\RequirePackage{thmtools}
\let\@xa\expandafter
\let\@nx\noexpand
\@ifundefined{c@thmt@dummyctr}{%
  \newcounter{thmt@dummyctr}%
  }{}
\gdef\theHthmt@dummyctr{dummy.\arabic{thmt@dummyctr}}%
\gdef\thethmt@dummyctr{}%
\long\def\thmt@collect@body#1#2\end#3{%
  \@xa\thmt@toks\@xa{\the\thmt@toks #2}%
  \def\thmttmpa{#3}%\def\thmttmpb{restatable}%
  \ifx\thmttmpa\@currenvir%thmttmpb
    \@xa\@firstoftwo% this is the end of the environment.
  \else
    \@xa\@secondoftwo% go on collecting
  \fi{% this is the end, my friend, drop the \end.
  % and call #1 with the collected body.
    \@xa#1\@xa{\the\thmt@toks}%
  }{% go on collecting
    \@xa\thmt@toks\@xa{\the\thmt@toks\end{#3}}%
    \thmt@collect@body{#1}%
  }%
}
\def\thmt@trivialref#1#2{%
  \ifcsname r@#1\endcsname
    \@xa\@xa\@xa\thmt@trivi@lr@f\csname r@#1\endcsname\relax\@nil
  \else #2\fi
}
\def\thmt@trivi@lr@f#1#2\@nil{#1}
\def\thmt@innercounters{%
  equation}
\def\thmt@counterformatters{%
  @alph,@Alph,@arabic,@roman,@Roman,@fnsymbol}

\@for\thmt@displ:=\thmt@counterformatters\do{%
  \@xa\let\csname thmt@\thmt@displ\@xa\endcsname\csname \thmt@displ\endcsname
}%
\def\thmt@sanitizethe#1{%
  \@for\thmt@displ:=\thmt@counterformatters\do{%
    \@xa\protected@edef\csname\thmt@displ\endcsname##1{%
      \@nx\ifx\@xa\@nx\csname c@#1\endcsname ##1%
        \@xa\protect\csname \thmt@displ\endcsname{##1}%
      \@nx\else
        \@nx\csname thmt@\thmt@displ\endcsname{##1}%
      \@nx\fi
    }%
  }%
  \expandafter\protected@edef\csname the#1\endcsname{\csname the#1\endcsname}%
  \ifcsname theH#1\endcsname
    \expandafter\protected@edef\csname theH#1\endcsname{\csname theH#1\endcsname}%
  \fi
}

\def\thmt@rst@storecounters#1{%
  \bgroup
        % ugly hack: save chapter,..subsection numbers
        % for equation numbers.
  %\refstepcounter{thmt@dummyctr}% why is this here?
  %% temporarily disabled, broke autorefname.
  \def\@currentlabel{}%
  \@for\thmt@ctr:=\thmt@innercounters\do{%
    \thmt@sanitizethe{\thmt@ctr}%
    \protected@edef\@currentlabel{%
      \@currentlabel
      \protect\def\@xa\protect\csname the\thmt@ctr\endcsname{%
        \csname the\thmt@ctr\endcsname}%
      \ifcsname theH\thmt@ctr\endcsname
        \protect\def\@xa\protect\csname theH\thmt@ctr\endcsname{%
          (restate \protect\theHthmt@dummyctr)\csname theH\thmt@ctr\endcsname}%
      \fi
      \protect\setcounter{\thmt@ctr}{\number\csname c@\thmt@ctr\endcsname}%
    }%
  }%
  \label{thmt@@#1@data}%
  \egroup
}%
\newif\ifthmt@thisistheone
\newenvironment{thmt@restatable}[3][]{%
  \thmt@toks{}% will hold body
  \stepcounter{thmt@dummyctr}% used for data storage label.
  \long\def\thmrst@store##1{%
    \@xa\gdef\csname #3\endcsname{%
      \@ifstar{%
        \thmt@thisistheonefalse\csname thmt@stored@#3\endcsname
      }{%
        \thmt@thisistheonetrue\csname thmt@stored@#3\endcsname
      }%
    }%
    \@xa\long\@xa\gdef\csname thmt@stored@#3\@xa\endcsname\@xa{%
      \begingroup
      \ifthmt@thisistheone
        % these are the valid numbers, store them for the other
        % occasions.
        \thmt@rst@storecounters{#3}%
      \else
        % this one should use other numbers...
        % first, fake the theorem number.
        \@xa\protected@edef\csname the#2\endcsname{%
          \thmt@trivialref{thmt@@#3}{??}}%
        % if the number wasn't there, have a "re-run to get labels right"
        % warning.
        \ifcsname r@thmt@@#3\endcsname\else
          \G@refundefinedtrue
        \fi
        % prevent stepcountering the theorem number,
        % but still, have some number for hyperref, just in case.
        \@xa\let\csname c@#2\endcsname=\c@thmt@dummyctr
        \@xa\let\csname theH#2\endcsname=\theHthmt@dummyctr
        % disable labeling.
        \let\label=\@gobble
        \let\ltx@label=\@gobble% amsmath needs this
        % We shall need to restore the counters at the end
        % of the environment, so we get
        % (4.2) [(3.1 from restate)] (4.3)
        \def\thmt@restorecounters{}%
        \@for\thmt@ctr:=\thmt@innercounters\do{%
          \protected@edef\thmt@restorecounters{%
            \thmt@restorecounters
            \protect\setcounter{\thmt@ctr}{\arabic{\thmt@ctr}}%
          }%
        }%
        % pull the new semi-static definition of \theequation et al.
        % from the aux file.
        \thmt@trivialref{thmt@@#3@data}{}%
      \fi
      % call the proper begin-env code, possibly with optional argument
      % (omit if stored via key-val)
      \ifthmt@restatethis
        \thmt@restatethisfalse
      \else
        \csname #2\@xa\endcsname\ifx\@nx#1\@nx\else[{#1}]\fi
      \fi
      \ifthmt@thisistheone
        % store a label so we can pick up the number later.
        \label{thmt@@#3}%
      \fi
      % this will be the collected body.
      ##1%
      \csname end#2\endcsname
      % if we faked the counter values, restore originals now.
      \ifthmt@thisistheone\else\thmt@restorecounters\fi
      \endgroup
    }% thmt@stored@#3
    % in either case, now call the just-created macro,
    \csname #3\@xa\endcsname\ifthmt@thisistheone\else*\fi
    % and artificially close the current environment.
    \@xa\end\@xa{\@currenvir}
  }% thm@rst@store
  \thmt@collect@body\thmrst@store
}{%
  %% now empty, just used as a marker.
}

\newenvironment{restatable}{%
  \thmt@thisistheonetrue\thmt@restatable
}{%
  \endthmt@restatable
}
\newenvironment{restatable*}{%
  \thmt@thisistheonefalse\thmt@restatable
}{%
  \endthmt@restatable
}

%%% support for keyval-style: restate=foobar
\protected@edef\thmt@thmuse@families{%
 \thmt@thmuse@families%
 ,restate phase 1%
 ,restate phase 2%
}
\newcommand\thmt@splitrestateargs[1][]{%
  \g@addto@macro\thmt@storedoptargs{,#1}%
  \def\tmp@a##1\@{\def\thmt@storename{##1}}%
  \tmp@a
}

\newif\ifthmt@restatethis
\define@key{restate phase 1}{restate}{%
  \thmt@thmuse@iskvtrue
  \def\thmt@storedoptargs{}% discard the first time around
  \thmt@splitrestateargs #1\@
  \def\thmt@storedoptargs{}% discard the first time around
  %\def\thmt@storename{#1}%
  \thmt@debug{we will restate as `\thmt@storename' with more args
  `\thmt@storedoptargs'}%
  \@namedef{thmt@unusedkey@restate}{}%
  % spurious "unused key" fixes itself once we are after tracknames...
  \thmt@restatethistrue
  \protected@edef\tmp@a{%
    \@nx\thmt@thisistheonetrue
    \@nx\def\@nx\@currenvir{\thmt@envname}%
    \@nx\@xa\@nx\thmt@restatable\@nx\@xa[\@nx\thmt@storedoptargs]%
      {\thmt@envname}{\thmt@storename}%
  }%
  \@xa\g@addto@macro\@xa\thmt@local@postheadhook\@xa{%
    \tmp@a
  }%
}
\thmt@mkignoringkeyhandler{restate phase 1}

\define@key{restate phase 2}{restate}{%
  % do not store restate as a key for repetition:
  % infinite loop.
  % instead, retain the added keyvals
  % overwriting thmt@storename should be safe here, it's been
  % xdefd into the postheadhook
  \thmt@splitrestateargs #1\@
}
\kv@set@family@handler{restate phase 2}{%
  \ifthmt@restatethis
  \@xa\@xa\@xa\g@addto@macro\@xa\@xa\@xa\thmt@storedoptargs\@xa\@xa\@xa{\@xa\@xa\@xa,%
    \@xa\kv@key\@xa=\kv@value}%
  \fi
}

\endinput
%%
%% End of file `thm-restate.sty'.

%%
%% This is file `thmtools.sty',
%% generated with the docstrip utility.
%%
%% The original source files were:
%%
%% thmtools.dtx  (with options: `thmtools')
%% This is a generated file.
%% 
%% This file is part of the `thmtools' package.
%% The `thmtools' package has the LPPL maintenance status: maintained.
%% Current Maintainer is Ulrich M. Schwarz, ulmi@absatzen.de
%% 
%% Copyright (C) 2008-2012 by Ulrich M. Schwarz.
%% 
%% This file may be distributed and/or modified under the
%% conditions of the LaTeX Project Public License, version 1.3a.
%% This version is obtainable at
%% http://www.latex-project.org/lppl/lppl-1-3a.txt
%% 
%% 
\NeedsTeXFormat {LaTeX2e}
\ProvidesPackage {thmtools}[2012/05/04 v63]
\DeclareOption{debug}{%
  \def\thmt@debug{\typeout}%
}
\let\@xa\expandafter
\let\@nx\noexpand
\def\thmt@debug{\@gobble}
\def\thmt@quark{\thmt@quark}
\newtoks\thmt@toks

\@for\thmt@opt:=lowercase,uppercase,anycase\do{%
  \@xa\DeclareOption\@xa{\thmt@opt}{%
    \@xa\PassOptionsToPackage\@xa{\CurrentOption}{thm-kv}%
  }%
}

\ProcessOptions\relax

\newcounter{thmt@dummyctr}%
\def\theHthmt@dummyctr{dummy.\arabic{thmt@dummyctr}}%
\def\thethmt@dummyctr{}%

\RequirePackage{thm-patch, thm-kv,
  thm-autoref, thm-listof,
  thm-restate}

\@ifpackageloaded{amsthm}{%
  \RequirePackage{thm-amsthm}
}{%
  \AtBeginDocument{%
  \@ifpackageloaded{amsthm}{%
    \PackageWarningNoLine{thmtools}{%
      amsthm loaded after thmtools
    }{}%
  }}%
}
\@ifpackageloaded{ntheorem}{%
  \RequirePackage{thm-ntheorem}
}{%
  \AtBeginDocument{%
  \@ifpackageloaded{ntheorem}{%
    \PackageWarningNoLine{thmtools}{%
      ntheorem loaded after thmtools
    }{}%
  }}%
}
\@ifclassloaded{beamer}{%
  \RequirePackage{thm-beamer}
}{}
\@ifclassloaded{llncs}{%
  \RequirePackage{thm-llncs}
}{}

\endinput
%%
%% End of file `thmtools.sty'.

%%
%% This is file `unique.sty',
%% generated with the docstrip utility.
%%
%% The original source files were:
%%
%% unique.dtx  (with options: `code')
%% This is a generated file.
%% 
%% This file is part of the `thmtools' package.
%% The `thmtools' package has the LPPL maintenance status: maintained.
%% Current Maintainer is Ulrich M. Schwarz, ulmi@absatzen.de
%% 
%% Copyright (C) 2008-2012 by Ulrich M. Schwarz.
%% 
%% This file may be distributed and/or modified under the
%% conditions of the LaTeX Project Public License, version 1.3a.
%% This version is obtainable at
%% http://www.latex-project.org/lppl/lppl-1-3a.txt
%% 
%% 
\NeedsTeXFormat {LaTeX2e}
\ProvidesPackage {unique}[2012/05/04 v63]

\DeclareOption{unq}{%
  \newwrite\uniq@channel
  \InputIfFileExists{\jobname.unq}{}{}%
  \immediate\openout\uniq@channel=\jobname.unq
  \AtEndDocument{%
    \immediate\closeout\uniq@channel%
  }
}
\DeclareOption{aux}{%
  \let\uniq@channel\@auxout
}

\newcommand\setuniqmark[1]{%
  \expandafter\ifx\csname uniq@now@#1\endcsname\relax
  \global\@namedef{uniq@now@#1}{\uniq@ONE}%
  \else
  \expandafter\ifx\csname uniq@now@#1\endcsname\uniq@MANY\else
  \immediate\write\uniq@channel{%
    \string\uniq@setmany{#1}%
  }%
  \ifuniq{#1}{%
    \uniq@warnnotunique{#1}%
  }{}%
  \fi
  \global\@namedef{uniq@now@#1}{\uniq@MANY}%
  \fi
}
\newcommand\ifuniq[1]{%
  \expandafter\ifx\csname uniq@last@#1\endcsname\uniq@MANY
  \expandafter \@secondoftwo
  \else
  \expandafter\@firstoftwo
  \fi
}
\def\uniq@ONE{\uniq@ONE}
\def\uniq@MANY{\uniq@MANY}
\newif\if@uniq@rerun
\def\uniq@setmany#1{%
  \global\@namedef{uniq@last@#1}{\uniq@MANY}%
  \AtEndDocument{%
    \uniq@warnifunique{#1}%
  }%
}
\def\uniq@warnifunique#1{%
  \expandafter\ifx\csname uniq@now@#1\endcsname\uniq@MANY\else
  \PackageWarningNoLine{uniq}{%
    `#1' is unique now.\MessageBreak
    Rerun LaTeX to pick up the change%
  }%
  \@uniq@reruntrue
  \fi
}
\def\uniq@warnnotunique#1{%
  \PackageWarningNoLine{uniq}{%
    `#1' is not unique anymore.\MessageBreak
    Rerun LaTeX to pick up the change%
  }%
  \@uniq@reruntrue
}
\def\uniq@maybesuggestrerun{%
  \if@uniq@rerun
  \PackageWarningNoLine{uniq}{%
    Uniquenesses have changed. \MessageBreak
    Rerun LaTeX to pick up the change%
  }%
  \fi
}
\AtEndDocument{%
  \immediate\write\@auxout{\string\uniq@maybesuggestrerun}%
}
\ExecuteOptions{aux}
\ProcessOptions\relax
\endinput
%%
%% End of file `unique.sty'.

\include{main-1.pdf}
\include{main-2.pdf}
\include{main-3.pdf}
\include{main-4.pdf}
\include{main-5.pdf}
\include{main-6.pdf}
\include{main-7.pdf}
\include{main-8.pdf}
\include{main-9.pdf}
\include{main-10.pdf}
\include{main-11.pdf}
\include{main-12.pdf}
\include{main-13.pdf}
\include{main-14.pdf}
\include{main-15.pdf}
\end{comment}

% Ubuntu, why the hell are you using 2009 TeX Live
\usepackage{gettitlestring}
\makeatletter
\def\NR@gettitle#1{%
	\GetTitleString{#1}%
	\let\@currentlabelname\GetTitleStringResult
}
\makeatother
\usepackage{thmtools}
\usepackage{thm-restate}

% Uncomment the following lines to hide all figures
%\usepackage{comment}
%\excludecomment{figure}
%\let\endfigure\relax

% RSI Specific macros
\numberwithin{equation}{section}
\theoremstyle{definition}
\newtheorem{op}{Operation}
\newcommand{\dobarbell}[2]{
	\vcenter{\hbox{\scalebox{#1}{#2}}}
}
\newcommand{\barbell}[1]{
	\mathchoice%
	{\dobarbell{1.6}{#1}}
	{\dobarbell{1.2}{#1}}
	{\dobarbell{0.9}{#1}}
	{\dobarbell{0.9}{#1}}
}
\def\ul#1{\underline{#1}}

\def\stimes{\otimes_{R^s}}
\def\ttimes{\otimes_{R^t}}

\def\MM{\mathcal M}
\def\AA{\mathcal A}
\def\SS{\mathcal S}
\def\BB{\mathfrak B}
%\usepackage{dsfont}
%\def\HH{\mathds H}
\def\HH{\mathcal H}
\DeclareMathOperator{\Top}{Top}


% For code
\usepackage{listings}
\lstset{basicstyle=\ttfamily,
	numbers=left,
	numbersep=5pt,
	numberstyle=\tiny,
	keywordstyle=\bfseries,
	title=\lstname,
	showstringspaces=false,
	frame=single}
\usepackage{paralist}

\usepackage{tikz}
\usepackage{pgfplots}
%\pgfplotsset{
%    every axis/.append style={scale only axis,width=0.85\textwidth,},
%}


% Small commands
\newcommand{\myqed}{\textsc{Q.e.d.}}
\newcommand{\cbrt}[1]{\sqrt[3]{#1}}
\newcommand{\floor}[1]{\left\lfloor #1 \right\rfloor}
\newcommand{\ceiling}[1]{\left\lceil #1 \right\rceil}
\newcommand{\mailto}[1]{\href{mailto:#1}{#1}}
\renewcommand{\iff}{\Leftrightarrow}
\renewcommand{\implies}{\Rightarrow}
\newcommand{\hrulebar}{
  \par\hspace{\fill}\rule{0.95\linewidth}{.7pt}\hspace{\fill}
  \par\nointerlineskip \vspace{\baselineskip}
}
\def\half{\frac{1}{2}}

%For use in author command
\newcommand{\plusemail}[1]{\\ \normalfont \texttt{\mailto{#1}}}

%More commands and math operators
\DeclareMathOperator{\cis}{cis}
\DeclareMathOperator{\lcm}{lcm}

%Convenient Environments
\newenvironment{soln}{\begin{proof}[Solution]}{\end{proof}}
\newenvironment{parlist}{\begin{inparaenum}[(i)]}{\end{inparaenum}}
\newenvironment{gobble}{\setbox\z@\vbox\bgroup}{\egroup}

%Inequalities
\newcommand{\cycsum}{\sum_{\text{cyc}}}
\newcommand{\symsum}{\sum_{\text{sym}}}
\newcommand{\cycprod}{\prod_{\text{cyc}}}
\newcommand{\symprod}{\prod_{\text{sym}}}

%From H113 "Introduction to Abstract Algebra" at UC Berkeley
\def\CC{\mathbb C}
\def\FF{\mathbb F}
\def\NN{\mathbb N}
\def\QQ{\mathbb Q}
\def\RR{\mathbb R}
\def\ZZ{\mathbb Z}
\newcommand{\normal}{\trianglelefteq}
\newcommand{\charin}{\text{ char }}
\DeclareMathOperator{\sign}{sign}
\DeclareMathOperator{\Aut}{Aut}
\DeclareMathOperator{\Inn}{Inn}
\DeclareMathOperator{\Syl}{Syl}

%From Kiran Kedlaya's "Geometry Unbound"
\def\abs#1{\lvert #1 \rvert}
\def\norm#1{\lVert #1 \rVert}
\def\dang{\measuredangle} %% Directed angle
\def\line#1{\overleftrightarrow{#1}}
\def\ray#1{\overrightarrow{#1}} 
\def\seg#1{\overline{#1}}
\def\arc#1{\wideparen{#1}}

%From M275 "Topology" at SJSU
\newcommand{\id}{\text{id}}
\newcommand{\taking}[1]{\stackrel{#1}{\longrightarrow}}
\newcommand{\inv}{^{-1}}

%From M170 "Introduction to Graph Theory" at SJSU
\DeclareMathOperator{\diam}{diam}
\DeclareMathOperator{\ord}{ord}
\newcommand{\defeq}{\stackrel{\text{def}}{=}}

%From the USAMO .tex filse
\def\st{^{\text{st}}}
\def\nd{^{\text{nd}}}
\def\rd{^{\text{rd}}}
\def\th{^{\text{th}}}
\def\dg{^\circ}
\def\be{\begin{enumerate}}
\def\bee{\begin{enumerate} \ii}
\def\ee{\end{enumerate}}
\def\bi{\begin{itemize}}
\def\bii{\begin{itemize} \ii}
\def\ei{\end{itemize}}
\def\ii{\item}

%Asy commands
\begin{asydef}
	import olympiad;
	import cse5;
	pointpen = black;
	pathpen = black;
	pathfontpen = black;
	anglepen = black;
	anglefontpen = black;

	// pen s = blue, t = red + dashed + 0.6;
	pen s = blue, t = red;
	pen dot_s = blue, dot_t = red;
\end{asydef}

\theoremstyle{plain}
\newtheorem{theorem}{Theorem}[section]
\newtheorem{lemma}[theorem]{Lemma} 
\newtheorem{conjecture}[theorem]{Conjecture} 
\newtheorem{corollary}[theorem]{Corollary}
\newtheorem{proposition}[theorem]{Proposition}

%Starred versions
\newtheorem*{theorem*}{Theorem}
\newtheorem*{lemma*}{Lemma}
\newtheorem*{conjecture*}{Conjecture}
\newtheorem*{corollary*}{Corollary}
\newtheorem*{proposition*}{Proposition}

%Def-style theorems
\theoremstyle{definition}
\newtheorem{claim}[theorem]{Claim}
\newtheorem{definition}[theorem]{Definition}
\newtheorem{fact}[theorem]{Fact}

\newtheorem{algorithm}{Algorithm}
\newtheorem{answer}{Answer}
\newtheorem{case}{Case}
\newtheorem{note}{Note}
\newtheorem{remark}{Remark}
\newtheorem{ques}{Question}

%Examples, exercises, problems
\newtheorem{example}{Example}
\newtheorem{exercise}{Exercise}
\newtheorem{problem}{Problem}
\newtheorem{joke}{Joke}

\newtheorem*{algorithm*}{Algorithm}
\newtheorem*{answer*}{Answer}
\newtheorem*{case*}{Case}
\newtheorem*{claim*}{Claim}
\newtheorem*{definition*}{Definition}
\newtheorem*{example*}{Example}
\newtheorem*{exercise*}{Exercise}
\newtheorem*{fact*}{Fact}
\newtheorem*{note*}{Note}
\newtheorem*{joke*}{Joke}
\newtheorem*{problem*}{Problem}
\newtheorem*{ques*}{Question}
\newtheorem*{remark*}{Remark}

